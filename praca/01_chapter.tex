\section{Płatności elektroniczne}


\subsection{Wprowadzenie}

E-commerce często utożsamia się tylko z dokonywaniem zakupów przez internet. Tymczasem występował już zdecydowanie wcześniej, z wykorzystaniem m.in. telefonu, faksu, czy telewizji. Odnosi się ogólnie do stosowania urządzeń elektronicznych w zakupie oraz sprzedaży. Jednak to właśnie internet jest dominującą obecnie formą e-handlu, będąc medium informacyjnym łączącym niejako wszystkie poprzednio używane. Jego nieograniczone możliwości przyciągają nowych użytkowników, którzy z czasem nabierają zaufania i stają się także kupującymi. W ten sposób dynamiczny rozwój sieci napędza także handel elektroniczny. Przedsiębiorcy chcąc zachować kontakt z klientami muszą zaznaczyć swoją obecność tam, gdzie koncentruje się większa część aktywności ludzkiej. Owocuje to powstawaniem nowych sklepów internetowych.

Płatności elektroniczne to bezgotówkowa formą zawierania transakcji. Są to wszelkiego rodzaju operacje finansowe dokonywane kanałami elektronicznymi. Możemy zaliczyć do nich m.in. karty płatnicze, czy polecenia przelewu. Bardzo zyskujące ostatnio na popularności są płatności internetowe, dokonywane za pośrednictwem sieci internetowej. Mogą występować pod różnymi postaciami, a należą do nich chociażby przelewy bankowe czy e-portmonetki. Transakcje takie zawierane są na odległość, za pośrednictwem urządzeń (instrumentów) elektronicznych - komputerów, tabletów lub smartfonów. Napotkać można się na nie w serwisach aukcyjnych, czy sklepach internetowych gdzie występują jako jedna z form zrealizowania opłaty za produkt lub usługę. 

Handel internetowy nie mógł by istnieć bez płatności internetowych. To właśnie wspólny rozwój e-handlu z internetem umożliwił powstanie nowej metody zawierania transakcji. Dzięki temu są one dobrze przystosowane do wymagań stawianych w rozwiązaniach z dziedziny e-commerce. Oferują one szybkość oraz wygodę w zawieraniu transakcji. Dzięki coraz większej konkurencji pomiędzy dostawcami usług płatniczych - także korzystniejsze prowizję. Ich zastosowanie rośnie, wraz z rosnącą liczbą usług oferowanych w internecie. Szczególnie zaznaczyły swoja obecność w aplikacjach mobilnych. Opłaty mogą być związane z uzyskaniem dostępu do takiego programu, bądź dodatkowej treści.

Po kilkunastu latach płatności elektroniczne dalej są w fazie dynamicznego rozwoju. Dzięki rozwojowi technologii znajdują się dla nich cały czas nowe zastosowania. 


\subsection{Ewolucja systemów płatniczych}

Płatności elektroniczne mają swój początek w e-bankowości. Wprowadzanie przez banki, a później instytucje pozabankowe, nowe udogodnienia technologiczne, spowodowały radykalną zmianę w sposobie przeprowadzania operacji finansowych. Przykładem tego mogą być karty płatnicze, zaprezentowane po raz pierwszy w latach pięćdziesiątych. Innym znaczącym osiągnięciem są pieniądze elektroniczne, także będące formą bezgotówkowych transakcji. 

\subsubsection*{Bankowość elektroniczna}

Bankowość elektroniczna kryje się pod wieloma nazwami: Internet banking, e-banking, on-line banking. Według J. Masiota jest to ``każda usługa bankowa, która umożliwia klientowi wzajemny kontakt z instytucją bankową z oddalonego miejsca poprzez: telefon, terminal, komputer osobisty, odbiornik telewizyjny z dekoderem'' \cite{pieniadz_elektroniczny-analiza}. Dodatkowo użytkownikowi oferowany jest podobny zakres usług jak w placówce fizycznej. Ogólnie dotyczy ona zdalnej obsługi konta bankowego. Pierwsze zastosowanie e-bankingu nastąpiło w USA, gdzie Diners Club wprowadził kartę płatniczą \cite{pieniadz_elektroniczny-analiza}. Nikt wtedy nie mógł zdawać sobie sprawy, jaką wielką popularność zyska ten instrument płatniczy. Następnie w 1970 r. powstał system kart debetowych, a w latach osiemdziesiątych pojawiły się  karty zawierające mikrochip. W Polsce pierwsze bankomaty powstały w 1990~r. za sprawą banku Pekao~S.A.

\subsubsection*{Bankowość internetowa}

Przed bankowością internetową istniały także inne formy e-bankingu, które pozwalały na odległość zarządzać swoim kontem bankowym. Jedną z takich form był home banking, który powstał głównie z myślą o klientach indywidualnych oraz małych przedsiębiorstwach. Po zainstalowaniu specjalnego oprogramowania, bądź kupienia odpowiedniej przystawki, klient mógł wykonywać operacje na swoim koncie. Ta i podobne odmiany e-bankingu nie zdążyły na dobre zaznaczyć swojej obecności. Wprowadzenie internetu do powszechnego użytku, przemodelowało dotychczas stosowane rozwiązania.

Początki sieci globalnej sięgają lat sześćdziesiątych XX stulecia, kiedy to na zlecenie Departamentu Obrony USA opracowany został ARPA-Net \cite{pieniadz_elektroniczny-analiza}. Od tego momentu Internet ewoluował, modyfikując stopniowo naszą rzeczywistość. Na przemiany społeczne wpłynęły przede wszystkim dogłębne zmiany w komunikowaniu się. Powstanie oraz rozwój sieci odbił się szczególnie na dziedzinach związanych z przetwarzaniem informacji \cite{pieniadz_elektroniczny-analiza}, czyli m.in. na sektor bankowy. Jego podatność na innowacje technologiczne pozwoliła na zupełnie nowy sposób dostępu do usług bankowych.

W bankowości internetowej oraz wirtualnej komunikacja odbywa się za pośrednictwem przeglądarki internetowej. Klient ma dostęp do większości usług oferowanych przez bank w placówce. Użytkownik może kontrolować stan konta, zaciągać kredyty lub wykonywać przelewy. Dostępność do takiej usługi jest niezależna od miejsca, 24 godziny na dobę i posiada wszystkie zalety, jakie niesie ze sobą korzystanie z internetu. La Jolla Bank FSB w 1994~r. był pierwszym bankiem, który umożliwił wykonywanie podstawowych operacji za pośrednictwem sieci. Ciekawą i dość popularną także w Polsce odmianą bankowości, jest bankowość wirtualna. Polega ona na obsłudze klienta tylko internetowo, a banki takie często nie posiadają nawet swoich placówek. Przykładem takich banków jest chociażby mBank. 

Sieć, by móc się rozprzestrzeniać, musiała przez lata wykształcić takie właściwości, jak: bezpieczeństwo, uniwersalność, interaktywność. Chcąc dokonać płatności, czy sprawdzić konto w banku chcemy mieć pewność, że nasze dane są bezpieczne. Istotny jest także sposób dostępu, coraz mniej zależny od używanego systemu operacyjnego. Na przestrzeni lat najważniejszy okazał się jednak stały rozwój. 

\subsubsection*{Pierwsza generacja płatności internetowych}

Rozpoczęta w latach dziewięćdziesiątych pierwsza generacja płatności, próbowała wprowadzić alternatywną gotówkę, np.: e-monety, czy tokeny \cite{elektroniczne_metody_platnosci}. E-gotówka miała zachować wszystkie cechy tradycyjnego pieniądza, oferując m.in. brak opłat transakcyjnych, czy anonimowość. Pierwszym systemem był wydany w 1994 r. e-cash, założony przez amerykańską firmę DigiCash. Podobnie jak w większości wprowadzanych w tamtym czasie rozwiązań, tak samo e-cash oznaczał elektroniczną walutę indywidualnym numerem seryjnym. Takie podejście miało chronić pieniądze przed fałszerstwem, a dostarczało tylko dodatkowych trudności. Cechą wspólną pierwszej generacji jest trudność w obsłudze oraz wymaganie dodatkowego oprogramowania, bądź nawet czytników kart. Sama firma DigiCash zakończyła swoją działalność w 1998 r.

\subsubsection*{Druga generacja płatności internetowych}

Trwająca do dziś i charakteryzująca się znacznie większą prostotą druga generacja, została zapoczątkowana na przełomie XX i XXI wieku. Jej powstanie i odmienność od wcześniej stosowanych rozwiązań, wynika z możliwości i ułatwień jakie posiada internet. Po kilkunastu latach dynamicznego rozwoju, zdążył się lepiej dostosować do stawianych mu wymagań. Szczególnie postęp do obszarze zabezpieczeń, wiążący się z powstaniem szyfrowanych protokołów przesyłania danych (np.: HTTPS), jest znaczący w systemach płatności. Nie są już potrzebne specjalne czytniki, wszystko może odbywać się przez przeglądarkę internetową. Sprawia to, że korzystanie z takiej formy płatności jest znaczenie wygodniejsze i szybsze, a także prostsze, gdyż zmniejsza się ilość kroków, jakie trzeba wykonać, aby dokonać zakupu. Lepsza edukacja oraz coraz dłuższe przebywanie w sieci sprawia, że ludzie częściej będą się decydować na tę formę płatności.  


\subsection{Analiza metod płatności}

Różnorodność dostępnych metod płatności internetowych sprawia, że mogą być one idealnie wpasowane w dany model biznesowy. Ważnym kryterium przy wyborze płatności jest wielkość pojedynczej transakcji w systemie, wiążąca się z poziomem zabezpieczeń. Na decyzję powinny także wpływać indywidualne preferencje użytkowników. Nie wypada kierować się tylko wygodą, czy innowacyjnością. Szczególnie istotny jest poziom zaufania, z jakim spotyka się dane rozwiązanie. Duża część użytkowników internetu przyzwyczajona jest do tradycyjnych płatności, szczególnie do gotówki i takiej formy zapłaty będą oczekiwać. Jest to ważny wybór, wpływający na odczucia płynące z korzystania z serwisu. Nie warto kierować się jedynie własnymi przekonaniami.

\subsubsection*{Wysokość transakcji}
Przedstawiony poniżej podział, dokonany został ze względu na wielkość pojedynczej transakcji. Obok każdej z kategorii zostały podane wartości, z którymi można się w ich przypadku najczęściej spotkać. Różnią się w zależności od dostawcy usług. Ich zadaniem jest jednie zobrazowanie jakich wielkości dotyczą. To rozróżnienie sugeruje przede wszystkim poziom zabezpieczeń, jaki należy zapewnić podczas przeprowadzania transakcji.
\begin{itemize}
	\renewcommand{\labelitemi}{--}
	\item Milipłatności - płatność do kilkudziesięciu groszy,
	\item Mikropłatności - 1~zł do 80~zł,
	\item Minipłatności - 80~zł do 800~zł,
	\item Makropłatności - wszystko powyżej 800~zł. 
\end{itemize}
Mikropłatności bardzo często dotyczą opłat za dobra niematerialne. Oprócz kwoty, dodatkowym ich wyróżnikiem jest krótki czas przeprowadzania płatności, czego spodziewa się użytkownik. Spotykane często w grach sieciowych, pozwalają na dokonanie transakcji bez np.: przerywania rozgrywki. Takie udogodnienie wpływa na poziom zabezpieczeń, który ma mniejsze znaczenie przy tego typu sumach. Ze względu na popularność, mikropłatności są bardzo ważne w rozwoju internetowych płatności. Wydawnictwa coraz częściej decydują się na cyfrową dystrybucję książek, czy artykułów. Wymienione wcześniej milipłatności zazwyczaj zaliczane są do mikropłatności.

Podejście do zabezpieczeń w przypadku minipłatności musi być zdecydowanie bardziej restrykcyjne, a dla makropłatności stanowi to swego rodzaju priorytet. Tego typu transakcje związane są z większą liczbą kroków, jaką konsument musi wykonać, aby transakcja mogła być zrealizowana. Najczęściej będzie się odbywała za pośrednictwem strony banku lub dostawcy usług płatności. Powoduje to, że zakupy są znacznie wolniejsze, jednak w tym przypadku nie jest to wadą. Dzięki  temu ryzyko niechcianych zakupów lub dokonania transakcji przez osobę trzecią jest mniejsze. Płatności mogą dotyczyć np.: zakupu sprzętu RTV lub AGD.

\subsubsection*{Moment pobrania}
Kolejnego rozróżnienia płatności można dokonać, ze względu na moment przelania pieniędzy z konta na konto. Decyduje o tym metody płatności, z jakiej korzystamy.
\begin{itemize}
	\renewcommand{\labelitemi}{--}
	\item System przedpłat (pay before) - użytkownik najpierw musi zasilić swoje wirtualne konto. Dopiero później ma możliwość dokonywania zakupów,
	\item System natychmiastowych płatności (pay now),
	\item System z odroczoną płatnością (pay later).
\end{itemize}

\subsubsection*{Metody płatności w internecie}
%TODO to chyba będzie dobre miejsce na obrazek z przepływem płatności
\begin{itemize}
	\renewcommand{\labelitemi}{--}
	\item Przelewy tradycyjne,
	\item Przelewy internetowe,
	\item Płatności komórką,
	\item Płatność portfelem internetowym,
	\item Płatność kartami płatniczymi.
\end{itemize}


\subsection{Bramki płatności online}


\subsection{Przyszłość}


\section{Płatności elektroniczne}

\subsection{Wprowadzenie}

% przejście od e-commerce do płatności, gdzie się z nimi spotykamy
E-commerce często utożsamia się tylko z dokonywaniem zakupów przez internet. Tymczasem występował już wcześniej pod wieloma postaciami i dotyczy m.in. telefonu, faksu, czy telewizji. Odnosi się ogólnie do stosowania urządzeń elektronicznych w zakupie oraz sprzedaży. Jednak to właśnie internet jest dominującą obecnie formą e-handlu, będąc medium informacyjnym łączącym niejako wszystkie poprzednio używane. Jego nieograniczone wręcz możliwości przyciągają nowych użytkowników, którzy z czasem nabierają zaufania i stają się także kupującymi. W ten sposób dynamiczny rozwój sieci napędza także handel elektroniczny. Przedsiębiorcy chcąc zachować kontakt z klientami muszą zaznaczyć swoją obecność tam, gdzie koncentruje się większa część aktywności ludzkiej. Owocuje to powstawaniem nowych sklepów internetowych.

% definicja e-płatności
Płatności elektroniczne to bezgotówkowa formą zawierania transakcji. Są to wszelkiego rodzaju operacje finansowe zawierane kanałami elektronicznymi. Możemy zaliczyć do nich m.in. karty płatnicze, czy polecenia przelewu. Bardzo zyskujące ostatnio na popularności są płatności internetowe, dokonywane za pośrednictwem sieci internetowej. Mogą być realizowane różnymi kanałami, a należą do nich chociażby płatności kartą płatniczą, przelewy bankowe czy e-portmonetki. Transakcje takie zawierane są na odległość, za pośrednictwem urządzeń (instrumentów) elektronicznych - komputerów, tabletów lub smartfonów. Napotkać można się na nie w serwisach aukcyjnych, czy sklepach internetowych gdzie występują jako jedna z form zrealizowania opłaty za produkt lub usługę. 

Handel internetowy nie mógł by istnieć bez płatności internetowych. To właśnie wspólny rozwój e-handlu z internetem umożliwił powstanie nowej metody zawierania transakcji. Dzięki temu są one dobrze przystosowane do wymagań stawianych w rozwiązaniach z dziedziny e-commerce. Oferują one szybkość oraz wygodę w zawieraniu transakcji. Dzięki coraz większej konkurencji pomiędzy dostawcami usług płatniczych - także korzystniejsze prowizję. Ich zastosowanie rośnie, wraz z rosnącą liczbą usług oferowanych w internecie. Szczególnie zaznaczyły swoja obecność w aplikacjach mobilnych. Opłaty mogą być związane z uzyskaniem dostępu do takiego programu, bądź dodatkowej treści.

% gdzie teraz sę e-płatności
Po kilkunastu latach płatności elektroniczne dalej są w fazie dynamicznego rozwoju. Dzięki rozwojowi technologii znajdują się dla nich cały czas nowe zastosowania. 

\subsection{Historia i etapy rozwoju}

% TODO Tu od początku o płatnościach - pieniądz elektroniczny, internet, bankowość. Bez opisu jak działają, tylko jak do nich doszło.

Powstanie płatności elektronicznych związane jest przede wszystkim z bankowością elektroniczną.

%TODO obszerny opis jak doszło do powstania płatności

\paragraph{Pierwsza generacja}

\paragraph{Druga generacja}

\subsection{Charakterystyka}

% TODO Wszelakie podziały - mikropłatności, bcb, rodzaje płatności. Także opis jak działają, co to jest - dokładniejsze od wprowadzenia, z danymi.
% Co to są e-płatności:
% - jakie mają formy - e-portmonetka, SMS, karty płatnicze, podział na internetowe
% - urządzenia elektroniczne
% - kanały realizacji
% - dostosowanie do internetu - dlaczego
% - PSP
% - jakie mają zalety - szybsze, płatności transgraniczne, niższa prowizja, 
% - o Polsce - że w Polsce słabo, ale będzie lepiej
% - porównanie z innymi

\noindent
\textbf{Rodzaje płatności}
\begin{itemize}
	\item Milipłatności,
	\item Mikropłatności,
	\item Minipłatności,
	\item Makropłatności. 
\end{itemize}

\noindent
\textbf{Moment pobrania}
\begin{itemize}
	\item System przedpłat (prepayment),
	\item System natychmiastowych płatności (pay-before),
	\item System z odroczoną płatnością (pay-later).
\end{itemize}

\noindent
\textbf{Metody płatności w internecie}
\begin{itemize}
	\item Przelewy tradycyjne,
	\item Przelewy internetowe,
	\item Płatności komórką,
	\item Płatność portfelem internetowym,
	\item Płatność kartami płatniczymi.
\end{itemize}

\noindent
\textbf{Typologia handlu elektronicznego}
\begin{itemize}
	\item Bezpośredni,
	\item Pośredni,
	\item Hybrydowy.
\end{itemize}

\noindent
\textbf{Ze względu na podmioty biorące udział}
\begin{itemize}
	\item B2B,
	\item B2C,
	\item C2C,
	\item C2B.
\end{itemize}

Korzyści:\\

Wady:\\


\subsection{Bramki płatności online}

% TODO A tutaj o PayPalu. Tu porównanie oraz jeszcze raz (jeśli było o nich wcześniej) opisać czym są - dokładnie.

\subsection{Przyszłość}

% TODO Perspektywy rozwoju na przyszłość. Ciągły wzrost popularności. Wszystko co z przyszłością. Też może się delikatnie 

Mobilne rozwiązania, płatność przez protokół GSM.

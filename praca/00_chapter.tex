\section*{Wstęp}
\addcontentsline{toc}{section}{Wstęp}

W największych polskich metropoliach powoli codziennością staje się korzystanie z systemów informatycznych, które swoim działaniem wspierają miejską infrastrukturę. Ich zastosowanie obejmuje
najczęściej transport publiczny, czy dotyczące tej pracy, strefy płatnego
parkowania, w których bilet może zostać kupiony przy użyciu aplikacji na urządzenie 
mobilne. Popularne stało się ostatnio wprowadzanie kart miejskich, takich jak 
Szczecińska Karta Aglomeracyjna, integrujących wybrane systemy miejskie. Pełnią 
one rolę wirtualnych portmonetek, przy użyciu których można zapłacić zarówno za 
bilet komunikacji, jak i wypożyczyć rower miejski. 

Wdrażanie podobnych rozwiązań nie byłoby możliwe, gdyby nie 
znaczący rozwój płatności elektronicznych. Dotyczą one 
realizowania transakcji finansowych przy użyciu elektronicznych instrumentów 
płatniczych. Przyczyniły się do powstania 
nowych form prowadzenia biznesu, takich jak e-commerce (ang.~handel elektroniczny), który wykorzystuje różne kanały elektroniczne (internet, faks, czy telewizję), do zawarcia transakcji handlowej. W ostatnich latach stał się jedną z ważniejszych gałęzi gospodarki 
\cite{barometr_radio}. Dynamiczny rozwój e-płatności 
wzajemnie napędzany był przez postęp technologiczny, a główny udział w tym miały 
zyskujące coraz większą popularność urządzenia mobilne. Szeroki zakres oferowanych w nich funkcjonalności 
dotyczy także płatności (tzw. m-płatności), a same smartfony bywają już 
zaliczane do instrumentów płatniczych. Najczęściej są wykorzystywane do 
niewielkich transakcji finansowych, tzw.~mikropłatności, jak chociażby zakup 
biletu postojowego.

Temat pracy został wybrany ze względu na chęć rozwinięcia już istniejących 
systemów płatności w strefie płatnego parkowania o kody QR. Realizacja zadań praktycznych wymaga zapoznania się 
z nowymi narzędziami oraz technologiami, takimi jak: tworzenie aplikacji 
internetowych oraz mobilnych i ich integracji z systemami płatności. Związana z 
tym możliwość zyskania nowych umiejętności też miała wpływ na wybór tematu.

Celem tej pracy jest stworzenie systemu płatności dla strefy płatnego 
parkowania, który umożliwia kupno i kontrolę biletu postojowego z 
wykorzystaniem urządzeń mobilnych oraz kodów QR.

Systemowi informatycznemu, który powstał w ramach tej pracy, została nadana 
nazwa ParQ. Najważniejsze zadania jakie należało zrealizować, 
dotyczyły implementacji części serwerowej oraz dwóch aplikacji mobilnych -- dla 
kontrolerów oraz kierowców. Interakcja użytkownika z systemem odbywa się 
jedynie za pośrednictwem urządzenia mobilnego, w którym ma on możliwość 
utworzenia konta, dodania pojazdu, czy zakupu biletu. Z każdym pojazdem 
powiązany jest specjalny numer identyfikacyjny, który następnie zostaje 
przedstawiony w postaci kodu QR. Do przeprowadzenia kontroli wystarczy 
zeskanowanie go za pomocą wbudowanego w telefon aparatu.

Cała praca została podzielona na pięć rozdziałów. W rozdziale 1. został 
poruszony temat płatności elektronicznych. Celem tej części jest wprowadzenie 
czytelnika do omawianego tematu. Pokrótce przedstawiono historię oraz etapy 
rozwoju, poddano analizie przyczyny coraz większej popularności e-płatności, a 
także ich wpływ na gospodarkę, czy modyfikację obecnych modeli biznesowych. 
Duża część rozdziału została poświęcona na zaprezentowanie różnych form 
płatności internetowych, razem z przedstawieniem ich wad oraz zalet. W końcowej 
części znajduje się opis zasad działania Strefy Płatnego Parkowania w 
Szczecinie, według których został stworzony system ParQ. Przedstawiono m.in. 
sposób w jaki naliczane są opłaty za postój. Omówione zostały także podobne 
systemy wspierające korzystanie ze stref parkowania, wykorzystujące aplikacje 
mobilne.

Rozdział 2. zawiera informacje o technologiach i narzędziach, które zostały 
użyte do realizacji celu pracy. Pierwszy podrozdział poświęcony jest kodom 
graficznym QR, używanych przy identyfikacji pojazdów w systemie. W tej części 
dokonano ich podziału, ze względu na wersje, typ przechowywanych danych oraz 
poziom korekcji błędów. Następna część rozdziału opisuje architekturę REST, 
której zalecenia zostały wykorzystane do komunikacji urządzeń mobilnych z 
serwerem w systemie. Dalszy fragment poświęcony jest systemowi Android, dla 
którego wykonana została część mobilna. Opisany tu został także  
framework Django, wykorzystany do implementacji aplikacji internetowej. Na końcu przedstawiony został PayPal, czyli usługodawca płatności online. 

W kolejnym 3. rozdziale zawarta została dokumentacja techniczna. Na początku przedstawiony został szczegółowy sposób działania systemu, a następnie wymieniono jego wymagania funkcjonalne i niefunkcjonalne. Kolejny podrozdział zawiera diagramy UML, w tym diagramy przypadków użycia, pakietów, klas, aktywności i sekwencji. Ostatnia część opisuje API (ang. Application Programming Interface), za pomocą którego aplikacja mobilna komunikuje się 
z serwerem, w tym zostały przedstawione także przykładowe zapytania oraz odpowiedzi.

Ostatni rozdział opisuje szczegóły implementacji systemu, z podziałem na część mobilną oraz serwerową. Zaprezentowano w nim fragmenty kodu, realizujące wybrane funkcjonalności. Dalsza część zawiera wyniki działania systemu oraz sposoby testowania, środowiska programistyczne i edytory tekstowe użyte do pisania kodu.

\section*{Podsumowanie}
\addcontentsline{toc}{section}{Podsumowanie}

Celem niniejszej pracy było stworzenie systemu płatności dla strefy płatnego parkowania. Do jego głównych zadań należało określenie wymagań systemu, zaimplementowanie serwera oraz aplikacji mobilnych. Użytkownicy korzystający z tego systemu mogą doładowywać swoje konta, rejestrować pojazdy oraz kupować bilety postojowe. Osoby kontrolujące mają możliwość sprawdzania ważności biletów postojowych. Główną cechą, która wyróżnia ten system spośród większości istniejących na rynku rozwiązań, jest zastosowanie kodów QR, za pomocą których identyfikowane są zaparkowane samochody. Postawiony w tej pracy cel został zrealizowany.
\\
\\
Pierwszy rozdział w większości poświęcony był płatnościom elektronicznym. W pierwszej części omówiona została Strefa Parkowania w Szczecinie oraz dokonano porównania systemu ParQ z już istniejącymi rozwiązaniami, które spełniają podobne zadania. Następnie dokonano analizy różnych metod płatności, pod kątem ich przydatności w systemach internetowych. Na bazie tych rozważań wybrana została forma zawierania transakcji, która najlepiej spełnia wymagania tworzonego systemu. W drugim rozdziale czytelnik został zapoznany z technologiami wykorzystywanymi do realizacji zadań pracy, a także pojęciami z nimi związanymi. Były one istotne dla pełnego zrozumienia dalszej części pracy. Kolejny rozdział zawiera dokumentację techniczną, natomiast ostatni, poświęcony został szczegółom implementacji.
\\
\\
Najbardziej pomocnymi pozycjami z bibliografii były publikacje o tematyce dotyczącej płatności elektronicznych. Szczególnie przydatna okazała się książka autorstwa Artura Borcucha, pt. \textit{Pieniądz elektroniczny pieniądz przyszłości -- analiza ekonomiczno-prawna}. W niezwykle wyczerpujący sposób opisane zostały w niej poszczególne etapy rozwoju pieniądza elektronicznego, które omówiono w rozdziale pierwszym. Szczególnie ważna na etapie implementacji systemu była dokumentacja Django, głównie dzięki przejrzystości oraz licznym przykładom w postaci fragmentów kodu, którymi opatrzone zostało prawie każde poruszane tam zagadnienie.
\\
\\
Mam nadzieję, że zebrane materiały okazały się wystarczające do zapoznania z zastosowanymi narzędziami oraz technologiami, użytymi do realizacji zadań pracy. Wybór tematu został podyktowany także rosnącą popularnością elektronicznych metod w internecie i aplikacjach mobilnych. Liczę, że podjęta analiza poruszonych tutaj zagadnień wykazała słuszność przyjętych rozwiązań. Praca nie wyczerpuje w pełni tematu, gdyż stworzony system ma szanse rozwoju w wielu kierunkach np.: wystawianie mandatu i rejestracja tego faktu na koncie użytkownika, powiadamianie kierowcy o zbliżającym się końcu opłaconego biletu.
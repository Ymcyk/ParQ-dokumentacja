\section{Płatności elektroniczne}


\subsection{Wprowadzenie}

% przejście od e-commerce do płatności, gdzie się z nimi spotykamy
E-commerce często utożsamia się tylko z dokonywaniem zakupów przez internet. Tymczasem występował już wcześniej pod wieloma postaciami i dotyczy m.in. telefonu, faksu, czy telewizji. Odnosi się ogólnie do stosowania urządzeń elektronicznych w zakupie oraz sprzedaży. Jednak to właśnie internet jest dominującą obecnie formą e-handlu, będąc medium informacyjnym łączącym niejako wszystkie poprzednio używane. Jego nieograniczone możliwości przyciągają nowych użytkowników, którzy z czasem nabierają zaufania i stają się także kupującymi. W ten sposób dynamiczny rozwój sieci napędza także handel elektroniczny. Przedsiębiorcy chcąc zachować kontakt z klientami muszą zaznaczyć swoją obecność tam, gdzie koncentruje się większa część aktywności ludzkiej. Owocuje to powstawaniem nowych sklepów internetowych.

% definicja e-płatności
Płatności elektroniczne to bezgotówkowa formą zawierania transakcji. Są to wszelkiego rodzaju operacje finansowe dokonywane kanałami elektronicznymi. Możemy zaliczyć do nich m.in. karty płatnicze, czy polecenia przelewu. Bardzo zyskujące ostatnio na popularności są płatności internetowe, dokonywane za pośrednictwem sieci internetowej. Mogą występować pod różnymi postaciami, a należą do nich chociażby przelewy bankowe czy e-portmonetki. Transakcje takie zawierane są na odległość, za pośrednictwem urządzeń (instrumentów) elektronicznych - komputerów, tabletów lub smartfonów. Napotkać można się na nie w serwisach aukcyjnych, czy sklepach internetowych gdzie występują jako jedna z form zrealizowania opłaty za produkt lub usługę. 

Handel internetowy nie mógł by istnieć bez płatności internetowych. To właśnie wspólny rozwój e-handlu z internetem umożliwił powstanie nowej metody zawierania transakcji. Dzięki temu są one dobrze przystosowane do wymagań stawianych w rozwiązaniach z dziedziny e-commerce. Oferują one szybkość oraz wygodę w zawieraniu transakcji. Dzięki coraz większej konkurencji pomiędzy dostawcami usług płatniczych - także korzystniejsze prowizję. Ich zastosowanie rośnie, wraz z rosnącą liczbą usług oferowanych w internecie. Szczególnie zaznaczyły swoja obecność w aplikacjach mobilnych. Opłaty mogą być związane z uzyskaniem dostępu do takiego programu, bądź dodatkowej treści.

% gdzie teraz sę e-płatności
Po kilkunastu latach płatności elektroniczne dalej są w fazie dynamicznego rozwoju. Dzięki rozwojowi technologii znajdują się dla nich cały czas nowe zastosowania. 


\subsection{Ewolucja systemów płatniczych}

Płatności elektroniczne mają swój początek w e-bankowości. Wprowadzanie przez banki, a później instytucje pozabankowe, nowe udogodnienia technologiczne, spowodowały radykalną zmianę w sposobie przeprowadzania operacji finansowych. Przykładem tego mogą być karty płatnicze, zaprezentowane po raz pierwszy w latach pięćdziesiątych. Innym znaczącym osiągnięciem są pieniądze elektroniczne, także będące formą bezgotówkowych transakcji. 

\subsubsection*{Bankowość elektroniczna}

Bankowość elektroniczna kryje się pod wieloma nazwami: Internet banking, e-banking, on-line banking. Według J. Masiota jest to ``każda usługa bankowa, która umożliwia klientowi wzajemny kontakt z instytucją bankową z oddalonego miejsca poprzez: telefon, terminal, komputer osobisty, odbiornik telewizyjny z dekoderem'' \cite{pieniadz_elektroniczny-analiza}. Dodatkowo użytkownikowi oferowany jest podobny zakres usług jak w placówce fizycznej. Ogólnie dotyczy ona zdalnej obsługi konta bankowego. Pierwsze zastosowanie e-bankingu nastąpiło w USA, gdzie Diners Club wprowadził kartę płatniczą \cite{pieniadz_elektroniczny-analiza}. Nikt wtedy nie mógł zdawać sobie sprawy, jaką wielką popularność zyska ten instrument płatniczy. Następnie w 1970 r. powstał system kart debetowych, a w latach osiemdziesiątych pojawiły się  karty zawierające mikrochip. W Polsce pierwsze bankomaty powstały w 1990~r. za sprawą banku Pekao~S.A.

\subsubsection*{Bankowość internetowa}

Przed bankowością internetową istniały także inne formy e-bankingu, które pozwalały na odległość zarządzać swoim kontem bankowym. Jedną z takich form był home banking, który powstał głównie z myślą o klientach indywidualnych oraz małych przedsiębiorstwach. Po zainstalowaniu specjalnego oprogramowania, bądź kupienia odpowiedniej przystawki, klient mógł wykonywać operacje na swoim koncie. Ta i podobne odmiany e-bankingu nie zdążyły na dobre zaznaczyć swojej obecności. Wprowadzenie internetu do powszechnego użytku, przemodelowało dotychczas stosowane rozwiązania.

Początki sieci globalnej sięgają lat sześćdziesiątych XX stulecia, kiedy to na zlecenie Departamentu Obrony USA opracowany został ARPA-Net \cite{pieniadz_elektroniczny-analiza}. Od tego momentu Internet ewoluował, modyfikując stopniowo naszą rzeczywistość. Na przemiany społeczne wpłynęły przede wszystkim dogłębne zmiany w komunikowaniu się. Powstanie oraz rozwój sieci odbił się szczególnie na dziedzinach związanych z przetwarzaniem informacji \cite{pieniadz_elektroniczny-analiza}, czyli m.in. na sektor bankowy. Jego podatność na innowacje technologiczne pozwoliła na zupełnie nowy sposób dostępu do usług bankowych.

W bankowości internetowej oraz wirtualnej komunikacja odbywa się za pośrednictwem przeglądarki internetowej. Klient ma dostęp do większości usług oferowanych przez bank w placówce. Użytkownik może kontrolować stan konta, zaciągać kredyty lub wykonywać przelewy. Dostępność do takiej usługi jest niezależna od miejsca, 24 godziny na dobę i posiada wszystkie zalety, jakie niesie ze sobą korzystanie z internetu. La Jolla Bank FSB w 1994~r. był pierwszym bankiem, który umożliwił wykonywanie podstawowych operacji za pośrednictwem sieci. Ciekawą i dość popularną także w Polsce odmianą bankowości, jest bankowość wirtualna. Polega ona na obsłudze klienta tylko internetowo, a banki takie często nie posiadają nawet swoich placówek. Przykładem takich banków jest chociażby mBank. 

Sieć, by móc się rozprzestrzeniać, musiała przez lata wykształcić takie właściwości, jak: bezpieczeństwo, uniwersalność, interaktywność. Chcąc dokonać płatności, czy sprawdzić konto w banku chcemy mieć pewność, że nasze dane są bezpieczne. Istotny jest także sposób dostępu, coraz mniej zależny od używanego systemu operacyjnego. Na przestrzeni lat najważniejszy okazał się jednak stały rozwój. 

\subsubsection*{Pierwsza generacja płatności internetowych}

Rozpoczęta w latach dziewięćdziesiątych pierwsza generacja płatności, próbowała wprowadzić alternatywną gotówkę, np.: e-monety, czy tokeny \cite{elektroniczne_metody_platnosci}. E-gotówka miała zachować wszystkie cechy tradycyjnego pieniądza, oferując m.in. brak opłat transakcyjnych, czy anonimowość. Pierwszym systemem był wydany w 1994 r. e-cash, założony przez amerykańską firmę DigiCash. Podobnie jak w większości wprowadzanych w tamtym czasie rozwiązań, tak samo e-cash oznaczał elektroniczną walutę indywidualnym numerem seryjnym. Takie podejście miało chronić pieniądze przed fałszerstwem, a dostarczało tylko dodatkowych trudności. Cechą wspólną pierwszej generacji jest trudność w obsłudze oraz wymaganie dodatkowego oprogramowania, bądź nawet czytników kart. Sama firma DigiCash zakończyła swoją działalność w 1998 r.

\subsubsection*{Druga generacja płatności internetowych}

Trwająca do dziś i charakteryzująca się znacznie większą prostotą druga generacja, została zapoczątkowana na przełomie XX i XXI wieku. Jej powstanie i odmienność od wcześniej stosowanych rozwiązań, wynika z możliwości i ułatwień jakie posiada internet. Po kilkunastu latach dynamicznego rozwoju, zdążył się lepiej dostosować do stawianych mu wymagań. Szczególnie postęp do obszarze zabezpieczeń, wiążący się z powstaniem szyfrowanych protokołów przesyłania danych (np.: HTTPS), jest znaczący w systemach płatności. Nie są już potrzebne specjalne czytniki, wszystko może odbywać się przez przeglądarkę internetową. Sprawia to, że korzystanie z takiej formy płatności jest znaczenie wygodniejsze i szybsze, a także prostsze, gdyż zmniejsza się ilość kroków, jakie trzeba wykonać, aby dokonać zakupu. Lepsza edukacja oraz coraz dłuższe przebywanie w sieci sprawia, że ludzie częściej będą się decydować na tę formę płatności.  

\subsection{Charakterystyka}

% TODO Wszelakie podziały - mikropłatności, bcb, rodzaje płatności. Także opis jak działają, co to jest - dokładniejsze od wprowadzenia, z danymi.

\subsubsection*{Wysokość transakcji}
\begin{itemize}
	\item Milipłatności,
	\item Mikropłatności,
	\item Minipłatności,
	\item Makropłatności. 
\end{itemize}

\subsubsection*{Moment pobrania}
\begin{itemize}
	\item System przedpłat (prepayment),
	\item System natychmiastowych płatności (pay-before),
	\item System z odroczoną płatnością (pay-later).
\end{itemize}

\subsubsection*{Metody płatności w internecie}
\begin{itemize}
	\item Przelewy tradycyjne,
	\item Przelewy internetowe,
	\item Płatności komórką,
	\item Płatność portfelem internetowym,
	\item Płatność kartami płatniczymi.
\end{itemize}

%\textbf{Ze względu na podmioty biorące udział}
%\begin{itemize}
%	\item B2B,
%	\item B2C,
%	\item C2C,
%	\item C2B.
%\end{itemize}


\subsection{Bramki płatności online}

% TODO A tutaj o PayPalu. Tu porównanie oraz jeszcze raz (jeśli było o nich wcześniej) opisać czym są - dokładnie.


\subsection{Przyszłość}

% TODO Perspektywy rozwoju na przyszłość. Ciągły wzrost popularności. Wszystko co z przyszłością. Też może się delikatnie 

Mobilne rozwiązania, płatność przez protokół GSM.

\section{Opłaty w strefie płatnego parkowania}

W tym rozdziale znajdują się informacje dotyczące szczecińskiej Strefy Płatnego Parkowania. Przedstawione są tutaj m.in. sposoby naliczania opłat, cennik, czy taryfikator. Informacje i wymagania związane z SPP są istotne dla tworzonego systemu, ponieważ to na ich bazie została zaimplementowana w nim logika biznesowa. W drugiej części tego rozdziału przedstawione są podobne rozwiązania do tworzonego systemu, które aktualnie funkcjonują w Szczecinie.

\subsection{Strefa Płatnego Parkowania w Szczecinie}

Zgodnie z obwieszczeniem Rady Miasta Szczecin z dnia 26 maja 2014~r. \cite{obwieszczenie} strefa płatnego parkowania w Szczecinie podzielona jest na dwie podstrefy: A i B. W każdej z nich obowiązuje inny cennik, który został przedstawiony w tabeli \ref{cennik}. W niej znajdują się także różne stawki, a to które z nich będą brane pod uwagę podczas naliczania opłaty, zależny od czasu postoju. Parkowanie w strefie jest płatne w dni robocze, w godzinach od 8:00 do 17:00 z wyjątkiem dni wolnych od pracy.

\begin{table}[h]
	\caption{Stawki w strefie płatnego parkowania w Szczecinie}
	\label{cennik}
	\begin{center}
		\begin{tabular}{| c | c | c | c |}
			\hline
			\multirow{2}{*}{L.p.} & \multirow{2}{*}{Opłaty jednorazowe} & \multicolumn{2}{| c |}{Kwota w zł.}\\
			\cline{3-4}
			&&Podstrefa A&Podstrefa B\\
			\hline
			1.&do 15 minut&0,70&0,40\\
			\hline
			2.&pierwsza godzina&2,80&1,60\\
			\hline
			3.&druga godzina&3,20&1,80\\
			\hline
			4.&trzecia godzina&3,60&2,00\\
			\hline
			5.&każda kolejna godzina&2,80&1,60\\
			\hline
		\end{tabular}
	\end{center}	
\end{table}

Ważna jest także informacja odnośnie czasu w którym bilety obwiązują, a mianowicie moment zakupu jest także chwilą, w której zaczynają być ważne. Nie ma możliwości zakupu biletów tzw. ``do przodu'', których czas rozpoczęcia jest późniejszy od daty zakupu \cite{sumowanie}. 

\subsection{Zakup i kontrola biletu}

Bilety postojowe są dostępne do kupienia w parkomatach, rozmieszczonych w miejscach gdzie obowiązuje strefa płatnego parkowania. Długość postoju ustalana jest na bazie kwoty, jaka została wpłacona oraz podstrefy, w której bilet został kupiony. Płatność może być realizowana gotówką, ale także przy użyciu karty SKA (Szczecińska Karta Aglomeracyjna), która w tym przypadku działa jak wirtualna portmonetka. Otrzymany wydruk potwierdzający opłacenie miejsca, należy umieścić za przednią szybą pojazdu. Kontroler sprawdza czy bilet został kupiony na odpowiednią podstrefę i czy nie przekroczono czasu postoju.

Oprócz parkomatów, od pewnego czasu w Szczecinie bilet może być kupiony także poprzez aplikację na urządzenia mobilne, a jednym z takich rozwiązań jest system moBiLET. Dostępny w wielu miastach Polski, pozwala na zakup nie tylko biletów postojowych, ale także komunikacji miejskiej, czy kolejowych. Po pobraniu aplikacji i zarejestrowaniu, niezbędne jest jeszcze doładowanie wirtualnej portmonetki, powiązanej z kontem użytkownika. W przypadku korzystania z strefy płatnego parkowania, konieczne jest jeszcze dodanie samochodu oraz umieszczenie w nim informacji, że bilet dla danego pojazdu został opłacony za pomocą tej właśnie aplikacji. Dzięki tej informacji, kontroler po wprowadzeniu numeru rejestracyjnego pojazdu, będzie mógł sprawdzić ważność biletu.

\subsubsection*{System ParQ}

Tworzony w ramach pracy system ParQ różni się od istniejących rozwiązań wykorzystaniem kodów QR. Każdy z pojazdów będzie musiał posiadać plakietkę z tym kodem, gdyż to za ich pomocą odbywa się identyfikacja pojazdów w systemie. Z perspektywy kierowcy, działanie tego systemu jest zbliżone do istniejących już rozwiązań. Po założeniu oraz doładowaniu konta odpowiednią kwotą, konieczne jest jeszcze dodanie swojego pojazdu. Po tym kroku kierowca otrzyma wiadomość e-mail z kodem QR, który powinien zostać umieszczony pod przednią szybą pojazdu. Od tego momentu wystarczy jedynie kupić bilet. 

Istotna różni się natomiast sposób przeprowadzania kontroli. Dzięki zastosowaniu kodów QR, nie ma potrzeby ręcznego wprowadzania numeru tablicy rejestracyjnej. Osoba sprawdzająca bilet będzie musiała jedynie zeskanować za pomocą aparatu wbudowanego w telefon plakietkę, znajdująca się pod przednią szybą pojazdu. Następnie informacja zwrócona z serwera tego systemu powiadomi go o tym, czy dany samochód ma opłacony bilet postojowy.
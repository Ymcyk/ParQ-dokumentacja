\section*{Wstęp}
\addcontentsline{toc}{section}{Wstęp}

Rozwijający się rynek elektroniczny doprowadził do powstania alternatywnych form biznesu. E-commerce (ang.~handel elektroniczny) \cite{biblia_ebiznesu} jest rozumiany jako całokształt prowadzenia działalności gospodarczej przez Internet \cite{pieniadz_elektroniczny-analiza}. W ostatnich latach stał się jedną z ważniejszych gałęzi gospodarki. Według prognoz jego wkład w PKB będzie się systematycznie zwiększać, uzupełniając lub zastępując tradycyjne metody sprzedaży. W Polsce wartość handlu elektronicznego w 2016~r. miała wynieść 35,8~mld.~zł. i jest to wzrost o 15\%, względem roku poprzedniego \cite{barometr_radio}. Powodami zwiększającego się znaczenia w gospodarce są zarówno nowi kupujący, jak i rosnąca liczba sklepów internetowych (e-sklepów) - 23,5~tys. w 2016~r \cite{barometr_radio}. Z drugiej strony dość niskie zaufanie oraz przyzwyczajenia konsumentów hamują rozpowszechnianie nowoczesnych metod handlu. Przykład pozostałych krajów Europy Zachodniej pokazuje, że gotówka będzie jednak coraz szybciej tracić na popularności \cite{pieniadz_elektroniczny-analiza}. Internetowe metody płatności są lepiej dostosowane do specyfiki e-commerce, oferując takie przewagi jak: szybkość, bezpieczeństwo czy wygodę.

Poziom nasycenia urządzeniami mobilnymi w krajach wysoko rozwiniętych często przekracza już 100\% \cite{biblia_ebiznesu}. Nierzadko jedna osoba używa różnych urządzeń w domu, pracy, czy podróży. Szeroki zakres oferowanych usług przekłada się równocześnie na zwiększanie wykorzystania internetu, gdzie większość ruchu jest generowana przez smartfony \cite{ruch_mobilny-internet}. To wpływa na zmiany społeczne oraz gospodarcze, w tym także e-biznes. Wraz z rozwojem technologicznym wykształcił się nowy rodzaj płatności, tzw. m-płatności \cite{biblia_ebiznesu}, gdzie urządzenia mobilne pełnią rolę instrumentów płatniczych. Najczęściej wykorzystywane są do niewielkich transakcji finansowych - mikropłatności (do ok. 80 zł \cite{elektroniczne_metody_platnosci}), do czego są najlepiej dostosowane, oferując możliwość szybkiej i wygodnej zapłaty. Często można się na nie natknąć w grach sieciowych, ale także innych aplikacjach mobilnych. Ta forma zyskuje ostatnio uznanie w przestrzeni miejskiej, gdzie bilet komunikacji publicznej można kupić właśnie internet, za pomocą telefonu. Lepszy kontakt z klientem to nie jest jedyna zaleta smartfonów. Wyposażone w aparat fotograficzny oraz coraz częściej w moduły GSM czy NFC, pozwalają na ``przenikanie'' świata wirtualnego z rzeczywistym. 

Celem tej pracy jest stworzenie systemu dla strefy płatnego parkowania, który będzie umożliwiał kupno oraz kontrolę biletu postojowego, z wykorzystaniem urządzeń mobilnych oraz kodu QR. W następujących akapitach zostanie przedstawiona tematyka dotycząca tej pracy.

Pierwszy rozdział został w całości poświęcony płatnością elektronicznym. 

\textbf{TODO: opis pozostałych rozdziałów pracy.}


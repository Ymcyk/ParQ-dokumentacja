\section*{Wstęp}
\addcontentsline{toc}{section}{Wstęp}

Rozwijający się rynek elektroniczny doprowadził do powstania alternatywnych form biznesu. E-commerce (pol.~handel elektroniczny) \cite{biblia_ebiznesu} jest rozumiany jako całokształt prowadzenia działalności gospodarczej za pośrednictwem różnych kanałów elektronicznych, głównie internetu \cite{pieniadz_elektroniczny-analiza}. W ostatnich latach stał się jedną z ważniejszych gałęzi gospodarki. Według prognoz jego wkład w PKB będzie się systematycznie zwiększać, uzupełniając lub zastępując tradycyjne metody sprzedaży. W Polsce wartość handlu elektronicznego w 2016~r. miała wynieść 35,8~mld.~zł. i jest to wzrost o 15\%, względem roku poprzedniego \cite{barometr_radio}. Powodami zwiększającego się znaczenia w gospodarce są zarówno nowi kupujący, jak i rosnąca liczba sklepów internetowych. Z drugiej strony dość niskie zaufanie oraz przyzwyczajenia konsumentów hamują rozpowszechnianie nowoczesnych metod handlu. Przykład pozostałych krajów Europy Zachodniej pokazuje, że gotówka będzie jednak coraz szybciej tracić na popularności \cite{pieniadz_elektroniczny-analiza}.

Poziom nasycenia urządzeniami mobilnymi w krajach wysoko rozwiniętych często przekracza już 100\% \cite{biblia_ebiznesu}. Nierzadko jedna osoba używa różnych urządzeń w domu, pracy, czy podróży. Szeroki zakres oferowanych usług przekłada się równocześnie na zwiększanie wykorzystania internetu, gdzie większość ruchu jest generowana przez smartfony \cite{ruch_mobilny-internet}. To wpływa na zmiany społeczne oraz gospodarcze, w tym także e-biznes. Wraz z rozwojem technologicznym wykształcił się nowy rodzaj płatności, tzw. m-płatności \cite{biblia_ebiznesu}, gdzie urządzenia mobilne pełnią rolę instrumentów płatniczych. Najczęściej wykorzystywane są do niewielkich transakcji finansowych - tzw.~mikropłatnościach, do czego są najlepiej dostosowane, oferując możliwość szybkiej i wygodnej zapłaty. Często można się na nie natknąć w grach sieciowych, ale także innych aplikacjach mobilnych. Lepszy kontakt z klientem to nie jest jedyna zaleta smartfonów. Wyposażone w aparat fotograficzny oraz coraz częściej w moduł NFC, pozwalają na ``przenikanie'' świata wirtualnego z rzeczywistym.

Postęp w dziedzinie płatności elektronicznych spowodował, że coraz częściej pojawiają się innowacyjne rozwiązania informatyczne w przestrzeni publicznej. W Szczecinie płatność elektroniczną portmonetką jest jedną z form zapłaty za przejazd komunikacją miejską. Także w innych obszarach gotówka jest powoli wypierana. Wybór tematu pracy był motywowany chęcią zapoznania się z płatnościami internetowymi oraz sposobami ich implementacji w systemach informatycznych.

Celem tej pracy jest stworzenie systemu dla strefy płatnego parkowania, który  umożliwia kupno oraz kontrolę biletu postojowego, z wykorzystaniem urządzeń mobilnych oraz kodów QR.

Z systemu użytkownik korzysta tylko za pośrednictwem aplikacji na urządzeniu mobilnym. Do kupienia biletu postojowego wymagane jest wcześniejsze zasilenie konta odpowiednimi środkami, oraz podanie numeru rejestracyjnego swojego samochodu. Każdy pojazd zarejestrowany w systemie posiada swój unikalny numer identyfikacyjny, znajdujący się na plakietce z kodem QR. To właśnie po zeskanowaniu tego kodu, kontroler otrzyma informacje czy dany pojazd ma wykupione miejsce postojowe.

W rozdziale 1 został poruszony temat płatności elektronicznych. Celem tej części jest wprowadzenie do omawianego zjawiska oraz zaprezentowanie jego skali. Pokrótce przedstawiono historię oraz etapy rozwoju. Poddano analizie przyczyny coraz większej popularności e-płatności, a także ich wpływ na gospodarkę, czy modyfikację obecnych modeli biznesowych. Duży nacisk został położony na zaprezentowanie różnych form płatności internetowych, razem z przedstawieniem wad oraz zalet. Na końcu opisane zostały bramki płatności online.

Drugi rozdział.

\textbf{TODO: opis pozostałych rozdziałów pracy.}


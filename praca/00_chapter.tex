\section*{Wstęp}
\addcontentsline{toc}{section}{Wstęp}

Pewną tendencją, która daje się zauważyć w największych polskich metropoliach, jest wprowadzanie różnych systemów informatycznych, mających za zadanie wspierać miejską infrastrukturę. Najczęściej zjawisko to występuje w przypadku transportu publicznego, czy dotyczących tej pracy stref płatnego parkowania, gdzie bilet może zostać kupiony przy użyciu aplikacji na urządzenie mobilne. Popularne jest także ostatnio wprowadzanie kart miejskich, takich jak Szczecińska Karta Aglomeracyjna, integrujących wybrane systemy miejskie. Pełnią one rolę wirtualnych portmonetek, przy użyciu których można zapłacić zarówno za bilet komunikacji oraz wypożyczyć rower miejski. 

Wdrażanie podobnych rozwiązań z pewnością jest bardzo wygodne dla ich użytkowników, jednak nie mogłoby do tego dojść, gdyby nie dwa czynniki, którymi są rozwój płatności elektronicznych i wzrost popularności urządzeń mobilnych. Pierwszy z nich dotyczy realizowania transakcji przy użyciu elektronicznych instrumentów płatniczych i znacząco przyczynił się do powstania alternatywnych form prowadzenia biznesu. E-commerce (ang.~handel elektroniczny) w ostatnich latach stał się jedną z ważniejszych gałęzi gospodarki \cite{barometr_radio}. Także dla płatności elektronicznych coraz większe znaczenie mają urządzenia mobilne. Ich popularność jest coraz większa, a poziom nasycenia w krajach wysoko rozwiniętych często przekracza już 100\% \cite{biblia_ebiznesu}. Szeroki zakres oferowanych funkcjonalności dotyczy także płatności (nazywanych m-płatnościami), a same smartfony bywają już nazywane instrumentami płatnicznymi. Najczęściej wykorzystywanych do niewielkich transakcji finansowych, tzw.~mikropłatności, jak chociażby zakup biletu postojowego. 

Temat pracy został wybrany, ze względu na chęć rozwinięcia już istniejących systemów płatności w strefie płatnego parkowania o kody QR, które mają posłużyć do identyfikacji pojazdów. Realizacja zadań praktycznych wymaga zapoznania się z nowymi narzędziami oraz technologiami, takimi jak: tworzenie aplikacji internetoych oraz mobilnych i ich intergracji z systemami płatności. Związana z tym możliwość zyskania nowych umiejętności też miała wpływ na podjęcie wyboru.

Celem tej pracy jest stworzenie systemu dla strefy płatnego parkowania, który  umożliwia kupno oraz kontrolę biletu postojowego, z wykorzystaniem urządzeń mobilnych oraz kodów QR.

Utworzony w ramach tej pracy system dla strefy płatnego parkowania, nazwany został ParQ. Zadania, które należało wykonać w jego zakresie, obejmowały napisanie aplikacji internetowej oraz osobnych aplikacji mobilnych dla kierowcy oraz kontrolera. Cała interakcja użytkownika z systemem odbywa się za pośrednictwem urządzenia mobilnego. Tam ma możliwość zarejestrowania, dodania pojazdu, czy zakupu biletu. Przed tym musi jednak zostać doładowane konto, z którym powiązana jest wirtualna portmonetka, co także realizowane jest z poziomu aplikacji mobilnej. Z każdym pojazdem powiązany jest numer identyfikacyjny UUID, który przedstawiony w postaci kodu QR zostaje następnie umieszczany za przednią szybą pojazdu. Kontroler skanując taką plakietkę aparatem urządzenia mobilnego z zainstalowaną aplikacją, zyskuje informację o ważności biletu.

W rozdziale 1 został poruszony temat płatności elektronicznych. Celem tej części jest wprowadzenie czytelnika do omawianego zjawiska. Pokrótce przedstawiono historię oraz etapy rozwoju, poddano analizie przyczyny coraz większej popularności e-płatności, a także ich wpływ na gospodarkę, czy modyfikację obecnych modeli biznesowych. Duży nacisk został położony na zaprezentowanie różnych form płatności internetowych, razem z przedstawieniem ich wad oraz zalet. Na końcu opisane są bramki płatności online.

Rozdział 2 zawiera informacje o technologiach i narzędziach, które zostały użyte do realizacji celu pracy. Pierwszy podrozdział poświęcony jest kodom graficznym QR, używanych przy identyfikacji pojazdów w systemie. Tam dokonano ich podziału, ze względu na wersje, typ przechowywanych danych oraz poziom korekcji błędów. Następna część rozdziału opisuje architekturę REST, której zalecenia zostały wykorzystane do komunikacji urządzeń mobilnych z serwerem w systemie. Dalszy fragment poświęcony jest systemowi Android. To właśnie dla niego wykonana została część mobilna pracy. Natomiast do implementacji aplikacji internetowej użyto frameworka Django, który opisany jest na końcu rozdziału. 

W rozdziale 3 znajdują się informacje na temat Strefy Płatnego Parkowania w Szczecinie, z którą system ParQ jest zgodny. Tutaj opisano sposób w jaki naliczane są opłaty za postój. Przedstawione zostały także podobne systemy do tworzonego w ramach tej pracy, które pozwalają na kupienie biletu za pośrednictwem urządzeń mobilnych.

W rozdziale 4 zawarta została cała dokumentacja techniczna. Na początku przedstawiony został sposób, w jaki system działa. Następnie wymienione są wymagania funkcjonalne oraz niefunkcjonalne. Kolejny podrozdział to diagramy UML, w tym diagramy przypadków użycia, pakietów, klas, aktywności i sekwencji. Ostatnia część opisuje api, za pomocą którego aplikacja mobilna komunikuje się z serwerem. Przedstawione zostały tam przykładowe zapytania oraz odpowiedzi.

Ostatni rozdział 5 opisuje szczegóły dotyczące implementacji systemu, z podziałem na część mobilną oraz serwerową. Zaprezentowano tam też fragmenty kodu, realizujące wybrane funkcjonalność. Dalsza część zawiera wyniki działania systemu, czyli zrzuty ekranu działających aplikacji mobilnych oraz internetowej. Tutaj opisane są też sposoby, jakimi testowano oprogramowanie oraz środowiska i edytory użyte do jego pisania.

\documentclass[12pt,a4paper]{article}

\usepackage[T1]{fontenc}
\usepackage[polish]{babel}
\usepackage[utf8]{inputenc}
\usepackage{lmodern}
% pakiet dodaje wcięcie do pierwszego paragrafu po
% tytule. Domyślnie pierwszy nie ma wcięcia.
\usepackage{indentfirst}
\usepackage{graphicx}
\usepackage[top=2.5cm, bottom=2.5cm, left=2.5cm, right=2.5cm]{geometry}
\usepackage{setspace}
\usepackage{hyperref}
\usepackage[parfill]{parskip}
\hypersetup{
	colorlinks,
	citecolor=black,
	filecolor=black,
	linkcolor=black,
	urlcolor=black	
}
\usepackage{times}
\setcounter{secnumdepth}{3}
\usepackage{pgfplots}
\graphicspath{{images/}}

\begin{document}
	\nocite{*}
	\pagenumbering{gobble}
	% strona tytułowa
	\begin{titlepage}
	% logo WI i ZUT
	\begin{figure}[!htb]
		\begin{minipage}{0.48\textwidth}
			\raggedright
			\includegraphics[width=0.8\linewidth]{ZUT_2.jpg}
		\end{minipage}\hfill
		\begin{minipage}{0.48\textwidth}
			\raggedleft
			\includegraphics[width=0.8\linewidth]{WI_2.jpg}
		\end{minipage}
	\end{figure}
	\vspace{2.5cm}
	\begin{onehalfspacing}
	\begin{center}
		\textbf{{\large Piotr Zieliński}}\\
		Nr albumu: 29979\\
		Kierunek studiów: Informatyka\\
		Specjalność: Techniki programowania\\
		Forma studiów: studia stacjonarne
	\end{center}
	\vspace{1.5cm}
	\begin{center}
		\textbf{\large SYSTEM PŁATNOŚCI W STREFIE PŁATNEGO PARKOWANIA Z WYKORZYSTANIEM URZĄDZEŃ MOBILNYCH ORAZ KODÓW QR}
	\end{center}
	\begin{center}
		\textbf{\large A PAYMENT SYSTEM IN A PAID PARKING ZONE WITH MOBILE DEVICES AND QR CODES}
	\end{center}
	\vspace*{\stretch{6}}
	\begin{center}
		praca dyplomowa inżynierska\\
		napisana pod kierunkiem:\\
		\textbf{{\large dr inż. Edwarda Półrolniczaka}}\\
		Katedra Systemów Multimedialnych
	\end{center}
	\vspace{1.5cm}
	%\noindent
	{\footnotesize Data wydania tematu pracy:}\\
	{\footnotesize Data złożenia pracy:}
	
	\vspace{.5cm}
	
	\end{onehalfspacing}
	
	\begin{center}
		Szczecin, 2017
	\end{center}

\end{titlepage}
	
	% pusta strona
	\mbox{}
	\thispagestyle{empty}
	\newpage
	
	\begin{onehalfspacing}
		% spis treści
		\pagenumbering{arabic}
		\setcounter{page}{3}
		\tableofcontents
		\newpage
		
		% wstęp
		\section*{Wstęp}
\addcontentsline{toc}{section}{Wstęp}

Pewną tendencją, która daje się zauważyć w największych polskich metropoliach, jest wprowadzanie różnych systemów informatycznych, mających za zadanie wspierać miejską infrastrukturę. Najczęściej zjawisko to występuje w przypadku transportu publicznego, czy dotyczących tej pracy stref płatnego parkowania, gdzie bilet może zostać kupiony przy użyciu aplikacji na urządzenie mobilne. Popularne jest także ostatnio wprowadzanie kart miejskich, takich jak Szczecińska Karta Aglomeracyjna, integrujących wybrane systemy miejskie. Pełnią one rolę wirtualnych portmonetek, przy użyciu których można zapłacić zarówno za bilet komunikacji oraz wypożyczyć rower miejski. 

Wdrażanie podobnych rozwiązań z pewnością jest bardzo wygodne dla ich użytkowników, jednak nie mogłoby do tego dojść, gdyby nie dwa czynniki, którymi są rozwój płatności elektronicznych i wzrost popularności urządzeń mobilnych. Pierwszy z nich dotyczy realizowania transakcji przy użyciu elektronicznych instrumentów płatniczych i znacząco przyczynił się do powstania alternatywnych form prowadzenia biznesu. E-commerce (ang.~handel elektroniczny) w ostatnich latach stał się jedną z ważniejszych gałęzi gospodarki \cite{barometr_radio}. Także dla płatności elektronicznych coraz większe znaczenie mają urządzenia mobilne. Ich popularność jest coraz większa, a poziom nasycenia w krajach wysoko rozwiniętych często przekracza już 100\% \cite{biblia_ebiznesu}. Szeroki zakres oferowanych funkcjonalności dotyczy także płatności (nazywanych m-płatnościami), a same smartfony bywają już nazywane instrumentami płatnicznymi. Najczęściej wykorzystywanych do niewielkich transakcji finansowych, tzw.~mikropłatności, jak chociażby zakup biletu postojowego. 

Temat pracy został wybrany, ze względu na chęć rozwinięcia już istniejących systemów płatności w strefie płatnego parkowania o kody QR, które mają posłużyć do identyfikacji pojazdów. Realizacja zadań praktycznych wymaga zapoznania się z nowymi narzędziami oraz technologiami, takimi jak: tworzenie aplikacji internetoych oraz mobilnych i ich intergracji z systemami płatności. Związana z tym możliwość zyskania nowych umiejętności też miała wpływ na podjęcie wyboru.

Celem tej pracy jest stworzenie systemu dla strefy płatnego parkowania, który  umożliwia kupno oraz kontrolę biletu postojowego, z wykorzystaniem urządzeń mobilnych oraz kodów QR.

Utworzony w ramach tej pracy system dla strefy płatnego parkowania, nazwany został ParQ. Zadania, które należało wykonać w jego zakresie, obejmowały napisanie aplikacji internetowej oraz osobnych aplikacji mobilnych dla kierowcy oraz kontrolera. Cała interakcja użytkownika z systemem odbywa się za pośrednictwem urządzenia mobilnego. Tam ma możliwość zarejestrowania, dodania pojazdu, czy zakupu biletu. Przed tym musi jednak zostać doładowane konto, z którym powiązana jest wirtualna portmonetka, co także realizowane jest z poziomu aplikacji mobilnej. Z każdym pojazdem powiązany jest numer identyfikacyjny UUID, który przedstawiony w postaci kodu QR zostaje następnie umieszczany za przednią szybą pojazdu. Kontroler skanując taką plakietkę aparatem urządzenia mobilnego z zainstalowaną aplikacją, zyskuje informację o ważności biletu.

W rozdziale 1 został poruszony temat płatności elektronicznych. Celem tej części jest wprowadzenie czytelnika do omawianego zjawiska. Pokrótce przedstawiono historię oraz etapy rozwoju, poddano analizie przyczyny coraz większej popularności e-płatności, a także ich wpływ na gospodarkę, czy modyfikację obecnych modeli biznesowych. Duży nacisk został położony na zaprezentowanie różnych form płatności internetowych, razem z przedstawieniem ich wad oraz zalet. Na końcu opisane są bramki płatności online.

Rozdział 2 zawiera informacje o technologiach i narzędziach, które zostały użyte do realizacji celu pracy. Pierwszy podrozdział poświęcony jest kodom graficznym QR, używanych przy identyfikacji pojazdów w systemie. Tam dokonano ich podziału, ze względu na wersje, typ przechowywanych danych oraz poziom korekcji błędów. Następna część rozdziału opisuje architekturę REST, której zalecenia zostały wykorzystane do komunikacji urządzeń mobilnych z serwerem w systemie. Dalszy fragment poświęcony jest systemowi Android. To właśnie dla niego wykonana została część mobilna pracy. Natomiast do implementacji aplikacji internetowej użyto frameworka Django, który opisany jest na końcu rozdziału. 

W rozdziale 3 znajdują się informacje na temat Strefy Płatnego Parkowania w Szczecinie, z którą system ParQ jest zgodny. Tutaj opisano sposób w jaki naliczane są opłaty za postój. Przedstawione zostały także podobne systemy do tworzonego w ramach tej pracy, które pozwalają na kupienie biletu za pośrednictwem urządzeń mobilnych.

W rozdziale 4 zawarta została cała dokumentacja techniczna. Na początku przedstawiony został sposób, w jaki system działa. Następnie wymienione są wymagania funkcjonalne oraz niefunkcjonalne. Kolejny podrozdział to diagramy UML, w tym diagramy przypadków użycia, pakietów, klas, aktywności i sekwencji. Ostatnia część opisuje api, za pomocą którego aplikacja mobilna komunikuje się z serwerem. Przedstawione zostały tam przykładowe zapytania oraz odpowiedzi.

Ostatni rozdział 5 opisuje szczegóły dotyczące implementacji systemu, z podziałem na część mobilną oraz serwerową. Zaprezentowano tam też fragmenty kodu, realizujące wybrane funkcjonalność. Dalsza część zawiera wyniki działania systemu, czyli zrzuty ekranu działających aplikacji mobilnych oraz internetowej. Tutaj opisane są też sposoby, jakimi testowano oprogramowanie oraz środowiska i edytory użyte do jego pisania.

		\newpage
		
		\section{Płatności elektroniczne}

\subsection{Wprowadzenie}

Ogólne informacje o płatnościach - czym są.

\subsection{Historia i etapy rozwoju}

Tu od początku o płatnościach - pieniądz elektroniczny, internet, bankowość.

\subsection{Charakterystyka}

Wszelaki podział - mikropłatności, bcb, rodzaje płatności.

\subsection{Bramki płatności online}

A tutaj o PayPalu.

		\newpage
		
		\section{Opis wykorzystanych technologii}
W tym rozdziale znajdują się informacje o najważniejszych technologiach, wykorzystanych podczas realizacji zadań pracy. W kolejnych podrozdziałach opisane zostały kody graficzne QR, system mobilny Android oraz framework aplikacji serwerowych - Django. 

\subsection{Kody graficzne QR}
Przedstawianie danych w postaci kodów graficznych nie jest niczym innowacyjnym - w sklepach towary oznaczane są za pomocą jednowymiarowego kodu kreskowego. Kombinacja jasnych oraz ciemnych linii umożliwia przechowywanie danych, które odczytywane są za pomocą skanera z laserem. Tego typu metody stosuje się głównie w celach identyfikacji. Do przechowywania większej ilości danych wykorzystuje się częściej tzw. kody 2D.

Kody QR (ang. Quick Respone - szybka odpowiedź) to dwuwymiarowe, kwadratowe kody graficzne. Składają się z modułów, czyli kombinacji ciemnych oraz jasnych kwadratów, które są nośnikami danych. Zostały stworzone przez japońską firmę Denso-Wave w 1994 r \cite{thonky_tutorial}. Według postanowień licencyjnych mogą być wykorzystywane bez żadnych opłat, a sam standard jest opisany w normie ISO/IEC 18004:2015 \cite{norma_qr}. Dzięki dodatkowemu wymiarowi, pozwalają na przechowywanie większej ilości informacji (do ok. 7000 liczb lub 4000 znaków alfanumerycznych) niż kody kreskowe, posiadające tylko jeden wymiar. Ponadto, zapewniają zdecydowanie lepszą korekcję błędów. Nawet częściowo uszkodzony kod może zostać poprawnie odczytany. Posiadają kilka miejsc szczególnych do ułatwienia orientacji podczas odkodowywania. Ich liczba zależy od rozmiaru kodu.

\begin{figure}[h]
	\begin{center}
		\includegraphics[width=0.2\textwidth]{02/qr_title}
	\end{center}
	\caption{Tytuł pracy przedstawiony w postaci kodu QR}
	\vspace{-0.3cm}
\end{figure}

Pierwotnie bardzo duże zastosowanie kody QR znalazły w logistyce, gdzie zawierały informacje o przesyłanych paczkach. Współcześnie kojarzone są przeważnie z urządzeniami mobilnymi. Spotykane na przystankach, w sklepach lub magazynach służą do komunikacji z użytkownikami smartfonów, przełamując niejako barierę między światem wirtualnym, a rzeczywistym. Kod zawiera informacje jedynie w postaci liczb, liter i symboli. Jednak odpowiednie formatowanie informacji, pozwala na dodatkowe interpretowanie ich przez urządzenie przenośne. I tak po zeskanowaniu może zostać wysłana wiadomość e-mail, albo dodany numer do kontaktów. Najczęściej jednak zawierają adresy URL, które powodują wyświetlenie odpowiedniej strony w przeglądarce internetowej telefonu.

\subsubsection*{Sposoby kodowania}
Przed zakodowaniem informacji do postaci kodu QR, należy określić trzy główne parametry. Są to: wersja, typ danych oraz poziom korekcji błędów.

Najważniejszym parametrem, wpływającym bezpośrednio na ilość danych jakie kod będzie w stanie przechować, to jego wersja. Numerowana jest od 1 do 40 i każda z nich ma przypisany do siebie rozmiar. Wersja pierwsza posiada 21 na 21 modułów, druga 24 na 24, a ostatnia, czyli czterdziesta - 177 na 177. Każda kolejna jest większa od poprzedniej o trzy moduły na bok. Oczywiście im numer wersji, czyli też rozmiar, jest większy, tym więcej danych będzie można zakodować. 

Równie ważne jak wybór wersji jest określenie typu danych, jaki ma być zakodowany. Informacja o typie zapisywana jest w kodzie QR, dzięki czemu podczas odczytywania czytnik wie jak ma interpretować dane. Dodatkowo, ta informacja decyduje też o maksymalnej pojemności. Typ numeryczny przechowuje informacje jedynie o liczbach, dlatego będzie potrzebował mniej bitów na znak, niż w przypadku danych alfanumerycznych. Dostępne są cztery typy:

\begin{itemize}
	\item Numeryczny -- ten tryb pozwala na zakodowanie tylko cyfr od 0 do 9, co umożliwia maksymalnie na przechowywanie 7089 znaków.
	\item Alfanumeryczny -- oprócz cyfr, także wielkie litery oraz znaki '\$', '\%', '*', '+', '-', '.', '/', ':' i spacja. Można zakodować do 4296 znaków. 
	\item Binarny -- domyślnie dla zestawu znaków z ISO-8859-1, ale także UTF-8. Maksymalnie 2953 znaków.
	\item Kanji -- znaki z systemu kodowania Shift JIS. Pomieści nie więcej niż 1817 znaków.
\end{itemize}

Poziom korekcji błędów służy do określenia, czy dane zostały odczytane poprawnie. Pozwala także na odzyskanie części z nich, nawet jeśli kod został uszkodzony (dzięki algorytmowi Reeda-Solomona). Specyfikacja wyróżnia cztery poziomy korekcji. Obok każdego z nich podany został procent danych, jakie można odzyskać:

\begin{itemize}
	\item L (Low) - 7\% danych,
	\item M (Medium) - 15\% danych,
	\item Q (Quartile) - 25\% danych,
	\item H (High) - 30\% danych.
\end{itemize}

\begin{table}[h]
	\caption{Pojemność kodów QR dla różnych ustawień}
	\vspace{0.3cm}
	\begin{center}
		\begin{tabular}{| c | c | c | c | c | c | c |}
			\hline
			Wersja & Moduły & Korekcja & Numeryczny & Alfanumeryczny & Binarny & Kanji\\
			\hline
			\multirow{4}{*}{1} & \multirow{4}{*}{21x21}&L&41&25&17&10\\
			& & M&34&20&14&8\\
			& & Q&27&16&11&7\\
			& & H&17&10&7&4\\
			\hline
			\multirow{4}{*}{40} & \multirow{4}{*}{177x177}&L&7089&4296&2953&1817\\
			& & M&5596&3391&2331&1435\\
			& & Q&3993&2420&1663&1024\\
			& & H&3057&1852&1273&784\\
			\hline
		\end{tabular}
	\end{center}
\end{table}

Tworzenie kodu graficznego QR jest procesem dość złożonym. Po analizie danych określającej ich typ, konieczne jest ich odpowiednie zakodowanie. Informacje dzielone są na bloki, do których dodawane są kolejne bity związane z korekcją błędów. Przed przedstawieniem danych w postaci modułów QR, ważne jest również odpowiednie ich maskowanie. Zbyt duża ilość kwadratów o tym samym kolorze w jednym miejscu, może spowodować błędne odczytanie. Dopiero po tych kilku etapach, może zostać wygenerowany kod. Praktycznie dla każdego języka istnieją biblioteki, które wykonują te wszystkie procesy. Wystarczy podać jedynie parametry kodu z danymi. Na przykład w Pythonie takim modułem jest \textit{qrcode}.

Odczytanie, czyli odkodowanie informacji możliwe jest na wiele sposobów. Razem z systemami mobilnymi często dostarczane są specjalne aplikacje, które wykorzystując wbudowaną kamerę, pozwalają na odczytanie zakodowanych danych. Podobnie, takie rozwiązanie możliwe jest dzięki odpowiednim biblioteką. W Androidzie jest to dostępna za darmo \textit{Zebra Crossing}.


\subsection{System Android}
% o systemach mobilnych. Procentowy udział - wykresik
% Android, krótko co to jest
% Android, krótka historia
% opisać jak to działa pod spodem. Java, kompilacja, Dalvik
Systemy na urządzenia mobilne z czasem stawały się coraz bardziej zaawansowane, przypominając swoją funkcjonalnością te przeznaczone na komputery. Dzisiaj oprócz obsługi podstawowych zadań telefonu, jak dzwonienie, pozwalają na przeglądanie internetu, czy instalowanie dodatkowych aplikacji. Do najpopularniejszych systemów w Polsce należą: Android z 65\% udziałem w rynku, Windows Phone - 16\% i iOS - 4\% \cite{polska_jest_mobi}.


\subsubsection*{O Androidzie}

Android to mobilna platforma systemowa, stworzona w 2003~r, a następnie wykupiona przez Google'a w 2005~r. z rąk niewielkiej firmy Android Inc. Od 2007~r. rozwijany jest w ramach sojuszu kilkudziesięciu firm - Open Handset Alliance. Android został zbudowany na bazie jądra Linuksa, i podobnie jak on rozpowszechniany jest za darmo z dostępnym publicznie kodem. To właśnie dostępność oraz możliwość dowolnego modyfikowania spowodowała, że zdołał w tak niedługim czasie zawładnąć rynkiem, stając się najpopularniejszym systemem mobilnym. Można go spotkać na większości popularnych urządzeń przenośnych, jak: telefony komórkowe, smartfony, tablety, netbooki. Jest stosowany także w e-bookach niektórych firm, czy innych sprzętach domowego użytku. 

%\begin{figure}[h]
%	\begin{center}
%		\includegraphics[width=0.1\textwidth]{02/android}
%	\end{center}
%	\caption{Logo Androida}
%\end{figure}

\subsubsection*{Architektura systemu}
Ze względu na architekturę systemu, można wyróżnić w Androidzie kilka abstrakcyjnych warstw: aplikacji, frameworku aplikacji, bibliotek, środowiska wykonawczego i jądra Linux, na którym bazuje cały system. Cała funkcjonalność systemu, niezbędna podczas działania aplikacji, dostępna jest poprzez framework aplikacji, czyli systemowe API napisane w Javie. Używając go, programista może kontrolować sposób działania oraz wygląd programu. Tutaj znajduje się menedżer aktywności (ang. Activity Manager), odpowiedzialny za cykl życia aplikacji, czy menedżer powiadomień (ang. Notification Manager) obsługujący wyświetlanie wszelkich notyfikacji. Także udostępnianie przez aplikacje interfejsów, z wykorzystaniem intencji, możliwe jest dzięki tej warstwie. Poniżej jej znajdują się natywne biblioteki, napisane w C i C++. Dzięki systemowemu API najczęściej nie ma konieczności z nich korzystać, i do tworzenia aplikacji można używać Javy. Najgłębiej w systemie znajduje się jego jądro, czyli Linux. Wykonuje ono najbardziej podstawowe funkcje, będąc odpowiedzialnym m.in. za zarządzanie baterią. Posiada też sterowniki systemowe: ekranu, aparatu, czy audio.

\begin{figure}[h]
	\begin{center}
		\includegraphics[width=0.5\linewidth]{02/warstwy_android}
	\end{center}
	\caption{Schemat architektury systemu Android}
\end{figure}

\subsubsection*{Środowisko wykonawcze}
Uruchamianiem programów napisanych w Javie zajmuje się wirtualna maszyna Javy (ang. Java Virtual Machine). Po skompilowaniu tworzony jest kod bajtowy, plik \textit{class}, który następnie po załadowaniu interpretowany jest przez JVM (możliwa też kompilacja JIT). Takie podejście pozwala na przenośność programów, czyli niezależność od platformy, kosztem pewnego spadku wydajności.

Mimo, że programy na Androida pisane są w Javie, to nie JVM używany jest do ich późniejszego wykonywana. Głównie jest to spowodowane chęcią stworzenia środowiska uruchomieniowego, które będzie lepiej przystosowane do słabszych wydajnościowo od komputerów maszyn, jakimi są urządzenia mobilne. Proces budowania aplikacji rozpoczyna się tak samo. Kompilator przekształca pliki zawierające kod Javy, do kodu bajtowego. W takiej postaci program nie mógłby zostać jeszcze uruchomiony. Najpierw specjalne narzędzie, pod nazwą \textit{dx}, modyfikuje wynik działania kompilatora. Dla każdej klasy Javy tworzony jest osobny plik \textit{class} - w trakcie działania programu są one ładowne przez JVM. Zadaniem \textit{dx} jest połączenie wszystkich tych kodów bajtowych w jeden plik \textit{dex}, z usunięciem powtarzających się symboli oraz zmianą znajdujących się tam instrukcji, na odpowiednie dla Androida. Dzięki temu skompilowany program będzie mniejszy oraz powinien działać szybciej. Ostatnim etapem budowania aplikacji jest stworzenie pliku \textit{apk}, odpowiednika \textit{jar}, w którym oprócz kodu bajtowego, znajdą się takie zasoby jak zdjęcia wykorzystywane w aplikacji.

\begin{figure}[h]
	\begin{center}
		\includegraphics[width=0.5\textwidth]{02/android_dex}
	\end{center}
	\caption{Proces budowy aplikacji}
	\vspace{-0.5cm}
\end{figure}

Do wersji 4.4 Androida (KitKat) aplikacje uruchamiane były w wirtualnej maszynie Dalvik. Sposób działania jest dość zbliżony do JVM. Kod bajtowy w postaci plików \textit{dex} jest interpretowany, z możliwością kompilacji do kodu natywnego (JIT). Jedną z różnic jest sposób działania wirtualnego procesora, który w Dalviku opartu został na rejestrach, a nie na stosie. Takie podejście wpływa na mniejsze zużycie pamięci, jednak programy są większe, gdyż instrukcje muszą zawierać dodatkowe informacje co do rejestrów, z których korzystają.

W wersji 5.0 Androida (Lolipop) zastąpiono dotychczasową maszynę wirtualną Dalvik, środowiskiem uruchomieniowym Android runtime (ART). Również przyjmuje pliki \textit{dex}, jednak nie interpretuje ich, a w momencie instalacji aplikacji - kompiluje. Taka kompilacja kodu pośredniego, języka wysokiego poziomu, do kodu natywnego, nosi nazwę Ahead-of-time (AOT). Przy każdym uruchomieniu aplikacji, dzięki ART, wykorzystywany jest jej natywny kod. Ta zmiana powoduje szybsze działanie i uruchamianie się aplikacji. Wadą jest znacznie dłuższy czas instalacji, który teraz obejmuje także dodatkową kompilację.


\subsubsection*{Programowanie aplikacji}
Aby móc tworzyć aplikacje na Androida z wykorzystaniem Javy, konieczne jest posiadanie:

\begin{itemize}
	\item Java z JDK i JRE,
	\item Android SDK.
\end{itemize}

W ten sposób aplikacja do funkcji systemowych odwoływać się będzie za pomocą udostępnionego przez system API, przeznaczonego dla Javy. Dzięki temu, że nie jest to kod natywny, nie jest konieczna osobna kompilacja dla każdej dostępnej architektury. Programy są uniwersalne i dopiero po zainstalowaniu na konkretnym urządzeniu, interpretowany jest kod bajtowy, bądź przeprowadzana kompilacja AOT. Przez twórców systemu udostępniane jest NDK, czyli Native Development Kit. Z jego pomocą aplikacje mogą być tworzone w C lub C++, odwołując się bezpośrednio do bibliotek systemowych. Niestety, mimo możliwego zysku na wydajności, stworzony kod jest zależny od architektury. Dodatkowo większość zewnętrznych bibliotek jest tworzonych w Javie. 

Bardzo zalecane jest używanie zintegrowanego środowiska programistycznego, które automatyzuje niektóre czynności. Potrafi stworzyć za użytkownika projekt z wymaganą strukturą plików, wykonać wszystkie etapy kompilacji, wgrać program na urządzenie i wiele innych. Dedykowanym IDE jest Android Studio, do niedawna wykorzystywany był domyślnie Eclipse.

\begin{figure}[h]
	\begin{center}
		\includegraphics[width=0.25\textwidth]{02/android_pliki}
	\end{center}
	\caption{Pliki projektu utworzonego w Android Studio}
	\vspace{-0.3cm}
\end{figure}

Podczas działania aplikacja prezentuje użytkownikowi tzw. ekrany. Są to odpowiedniki okien systemowych, gdzie umieszczane są elementy graficznego interfejsu użytkownika, czyli widoki (ang. views), jak np.: przyciski, rozwijane listy, czy pola tekstowe. Wchodząc z nimi w interakcję (poprzez dotyk, w ekranach dotykowych), możliwa jest komunikacja między użytkownikiem, a urządzeniem. 

Wygląd ekranów definiowany jest w plikach XML, nazywanych układami (ang. layout). Każdy z widoków jest osobnym znacznikiem, a za pomocą argumentów można modyfikować wybrane parametry jak rozmiar, czy kolor. Wszystkie widoki w pliku XML muszą posiadać swojego rodzica, który definiuje jak mają one być traktowane w tym układzie. W Androidzie można skorzystać z trzech. RelativeLayout - położenie widoków określane jest względem siebie. LinearLayout - układ liniowy, gdzie elementy GUI wyświetlane są jeden koło drugiego. GridLayout - dzieli ekran na siatkę, składającą się z wierszy oraz kolumn, i pozwala umieścić widoki we wskazanych komórkach.

Układy definiują jak dany ekran ma wyglądać, natomiast aktywności (ang. activity), czyli klasy Javy, określają w jaki sposób mają one reagować na interakcje użytkownika. Przechodząc do danego ekranu, tworzona jest najpierw aktywność. W metodzie onCreate, wywoływanej przed wyświetleniem ekranu, wybierany jest układ, który ma zostać użyty przez system do stworzenia interfejsu. Widoki z tego układu mogą mieć powiązane ze sobą jakieś operacje, które będą przeprowadzane np.: po naciśnięciu przycisku. To także odbywa się w aktywności, która może posiadać metody wywoływane po zakończeniu danej interakcji.

\subsubsection*{Podsumowanie}
Rozwiązania mobilne cieszą się coraz większym zainteresowaniem. Tylko w sklepie z aplikacjami Androida, Google Play, liczba programów przekroczyła w 2015 r. 1,6 mln \cite{biblia_ebiznesu_2}. Powstające coraz to nowe urządzenia, na których zainstalowany jest Android sprawia, że umiejętność programowania na tą platformę będzie coraz bardziej doceniana.


\subsection{Framework Django}
Frameworki aplikacji internetowych powstały z myślą o zwolnieniu programisty z obowiązku pisania części kodu, który jest wspólny dla większości serwisów. Może to być dostęp do bazy danych, obsługa linków (ang.~routing), czy zarządzanie sesjami. Dodatkowo, pisanie aplikacji internetowej od podstaw, jest zadaniem czasochłonnym oraz dość trudnym. Frameworki są odpowiedzią na te problemy, dostarczając zestaw gotowych oraz przetestowanych rozwiązań, które należy dostosować do własnych potrzeb. Umożliwiają utworzenie struktury plików projektu, na bazie którego dalej będzie rozwijana aplikacja. Każdy z najpopularniejszych języków programowania oferuje duży wybór silników, przeznaczonych do tworzenia usług internetowych i np.: w C\# napisane zostały ASP.NET i MonoRail, w Javie - Spring i JavaServer Faces, a w Python - Django i Flask. Mogą się różnić przede wszystkim stopniem złożoności, dzięki czemu nadają się do różnych zastosowań - prezentują odmienne sposoby realizacji podobnych zadań. Wybór programisty będzie głównie zależny od jego prywatnych preferencji.

\subsubsection*{Początki Django}
Django rozwijany jest jako wolne oprogramowanie na GitHub'ie, w ramach fundacji Django Software Foundation, ale skupiaja wokół siebie także wielu niezależnych twórców. Został napisany w Pythonie, stając się z czasem najpopularniejszym frameworkiem dla tego języka. Jego historia rozpoczęła się w 2003~r., kiedy to dwóch programistów Adrian Holovaty i Simon Willison zaczęli używać Pythona do tworzenia aplikacji webowych dla kilku serwisów informacyjnych, m.in. Lawrence.com. Praca w środowisku dziennikarskim, cechująca się napiętym grafikiem, wymagała niezwykle szybkiej realizacji zadań. Z tej konieczności opracowali własny silnik, który w 2005~r., już pod nazwą Django, został udostępniony publicznie. To właśnie szybkość oraz względna łatwość tworzenia aplikacji są jego głównymi zaletami.

\subsubsection*{Charakterystyka}
% trochę upośledzone MVC
% diagram 
% zabezpieczenia
% admin
% modele
% o dodatkach parę słów
Aplikacje Django tworzone są w interpretowanym języku Python, przeznaczonym głównie do pisania skryptów. Jako że jego interpretery dostępne są na wielu platformach, jest on niezależny od systemu operacyjnego. Dodatkowo wsparcie do wielu paradygmatów (imperatywnego, funkcyjnego, obiektowego), przekłada się na jego szerokie zastosowanie. Oprócz programowania serwerów, jest używany do pisania testów, aplikacji z graficznym interfejsem, a także gier 3D. Wyróżnia się głównie charakterystyczną składnią, w której poszczególne bloki kodu odseparowane są wcięciami (spacje lub tabulator). Wpływa to na zwiększoną czytelność kodu.

Architektura aplikacji sieciowych typu klient-serwer, opiera się często na wzorcu projektowym Model-View-Controller (pol.~Model-Widok-Kontroler), w skrócie MVC. Jego ideą jest odseparowanie części prezentacji danych, od kodu odpowiedzialnego za ich przetwarzanie. Aplikacja dzielona jest w nim na trzy główne części. Model jest reprezentacją danych oraz logiki problemu. Widok odpowiedzialny jest za prezentację - określa w jaki sposób informacje mają zostać przedstawione. Kontroler przyjmuje żądania i wykonuje związane z nimi akcje. Głównie aktualizuje widoki oraz modele.

\begin{figure}[h]
	\begin{center}
		\includegraphics[width=0.6\textwidth]{02/MVC_diagram}
	\end{center}
	\caption{Wzorzec projektowy MVC}
	\vspace{-0.3cm}
\end{figure}

Wzorzec MVC jest stosowany w Django, jednak został zrealizowany w sposób odmienny od najczęściej spotykanych jego implementacji. Z tego względu często nazywa się go wzorcem MTV (Model-Template-View), będącego pewną wariacją MVC. Także wyróżnia się w nim trzy główne części aplikacji, a są to:
\begin{itemize}
	\item Model -- podobnie jak w MVC, zapewnia dostęp do danych. Opisane są tutaj relacje między danymi oraz odbywa się ich walidacja.
	\item Template (pol. szablon) -- warstwa prezentacji, czyli jak ma zostać wygenerowana odpowiedź (w postaci dokumentu HTML lub innym formacie).
	\item View (pol. widok) -- zawiera logikę biznesową. Pobiera dane z modeli i łączy je z szablonami, tworząc w ten sposób odpowiedź. Stanowi pomost między dwoma wcześniej wymienionymi elementami MTV.
\end{itemize}
Rolę kontrolera MVC pełni w Django sam framework. Otrzymane zapytanie wysyłane jest do odpowiedniego widoku, w zależności od konfiguracji URL.

\begin{singlespace}
	\captionof{listing}{Przykładowa konfiguracja URL w Django}
	\vspace{0.3cm}
	\inputminted[fontsize=\footnotesize]{python}{src/urls.py}
	\label{l:url}
\end{singlespace}

% tutaj o wszystkich cechach Django
W przeciwieństwie do mikro frameworków, takich jak Flask, Django dostarcza programiście wielu gotowych rozwiązań, jak: dostęp do baz danych, klasy ORM, obsługa linków URL, uprawnienia użytkowników, automatycznie generowana strona administratora, szablony HTML i wiele innych. Dodatkowo posiada wbudowaną ochronę przed typowymi atakami, takimi jak: cross-site scripting (XSS), cross-site request forgery (CSRF), a także wstrzykiwanie SQL'a.

\subsubsection*{Programowanie aplikacji}
Tak jak większość frameworków, także i Django tworzy za użytkownika gotową strukturę projektu. Poza konfiguracją, nie znajduje się w nim jednak żadna logika. Wszystkie modele oraz widoki w danym projekcie tworzone są w ramach tzw. aplikacji, czyli pakietów Pythona (których struktura także generowana jest przez framework), odpowiednio w plikach models.py oraz views.py. Dzięki temu mogą zostać one ponownie wykorzystane. W hierarchii plików znajdują się na tym samym poziomie, co projekt. 

\begin{figure}[h]
	\begin{center}
		\includegraphics[width=0.15\textwidth]{02/django_structuree}
	\end{center}
	\caption{Struktura plików projektu (ParQ) z aplikacją (badges)}
	\vspace{-0.3cm}
\end{figure}

W pliku settings.py projektu znajdują się wszystkie ustawienia serwisu, takie jak konfiguracja bazy danych, strefy czasowej, silnika szablonów oraz pakietów pośredniczących (middleware) wykorzystywanych np.: do autoryzacji. Tam także powinny zostać zarejestrowane wszystkie aplikacje używane w projekcie.

W models.py aplikacji znajdują się modele danych - są to klasy dziedziczące po Model. Posiadają zmienne klasowe, które reprezentują kolumny w tabelach baz danych. Nazwa takiego pola jest później używana jako nazwa kolumny, natomiast jej typ definiowany jest przez instancję jednej z klas pochodnych od Field. I tak dla przykładu IntegerField będzie typem całkowitoliczbowym, a CharField znakowym. Parametry podane w konstruktorze pozwalają zdefiniować rozmiar, czy unikalność krotki w kolumnie. Relacje również tworzone są za pomocą pól. Model posiadający pole relacji, może odwoływać się za jego pomocą do powiązanych danych. W Django znajdują się trzy takie klasy:

\begin{itemize}
	\item ForeignKey -- pole z kluczem obcym, używane w relacjach jeden do wielu. Tworzona jest kolumna w tabeli.
	\item ManyToManyField -- używane w relacjach wiele do wielu. W bazie danych zostanie automatycznie utworzona tabela pośrednia, z kluczami powiązanych tabel.
	\item OneToOneField -- do relacji jeden do jeden. Także wykorzystywana jest tabela pośrednia, jednakże oba klucze są unikalne - mogą wystąpić tylko w jednej relacji w ramach tej tabeli.
\end{itemize}

W modelach dozwolone jest także definiowanie własnych metod.

\begin{singlespace}
	\captionof{listing}{Przykładowy model danych z modułu models.py}
	\vspace{0.3cm}
	\inputminted[fontsize=\footnotesize]{python}{src/models.py}
\end{singlespace}

Kolejnym ważnym plikiem w aplikacji tworzonej przez użytkownika jest views.py. To tutaj znajdują się widoki ze wzorca MTV i scalają one szablony z modelami. Jako parametr przyjmują obiekt HttpRequest, zawierający dane zawarte w żądaniu HTTP. To właśnie te widoki podawane są podczas konfiguracji linków w funkcji url, pliku urls.py w projekcie. Zostało to zaprezentowane na listingu \ref{l:url}.

\begin{singlespace}
	\captionof{listing}{Widok}
	\vspace{0.3cm}
	\inputminted[fontsize=\footnotesize]{python}{src/views.py}
\end{singlespace}

\subsubsection*{Aplikacje jako dodatki}
Napisane aplikacje Django, mogą być wielokrotnie wykorzystywane w innych projektach. Na tej zasadzie funkcjonują dodatki pisane do tego frameworku. Jednymi z nich są: Django REST Framework używany do tworzenia API w architekturze REST, czy django-annoying modyfikujący działanie niektórych elementów silnika.

%\begin{listing}[H]
%	\caption{Kod powyżej}
%	\vspace{0.5cm}
%	\inputminted[fontsize=\footnotesize]{python}{src/models.py}
%\end{listing}

		\newpage
		
		\section{Opłaty w strefie płatnego parkowania}

W tym rozdziale znajdują się informacje dotyczące szczecińskiej Strefy Płatnego Parkowania. Przedstawione są tutaj m.in. sposoby naliczania opłat, cennik, czy taryfikator. Informacje i wymagania związane z SPP są istotne dla tworzonego systemu, ponieważ to na ich bazie została zaimplementowana w nim logika biznesowa. W drugiej części tego rozdziału przedstawione są podobne rozwiązania do tworzonego systemu, które aktualnie funkcjonują w Szczecinie.

\subsection{Strefa Płatnego Parkowania w Szczecinie}

Zgodnie z obwieszczeniem Rady Miasta Szczecin z dnia 26 maja 2014~r. \cite{obwieszczenie} strefa płatnego parkowania w Szczecinie podzielona jest na dwie podstrefy: A i B. W każdej z nich obowiązuje inny cennik, który został przedstawiony w tabeli \ref{cennik}. W niej znajdują się także różne stawki, a to które z nich będą brane pod uwagę podczas naliczania opłaty, zależny od czasu postoju. Parkowanie w strefie jest płatne w dni robocze, w godzinach od 8:00 do 17:00 z wyjątkiem dni wolnych od pracy.

\begin{table}[h]
	\caption{Stawki w strefie płatnego parkowania w Szczecinie.}
	\label{cennik}
	\begin{center}
		\begin{tabular}{| c | c | c | c |}
			\hline
			\multirow{2}{*}{L.p.} & \multirow{2}{*}{Opłaty jednorazowe} & \multicolumn{2}{| c |}{Kwota w zł.}\\
			\cline{3-4}
			&&Podstrefa A&Podstrefa B\\
			\hline
			1.&do 15 minut&0,70&0,40\\
			\hline
			2.&pierwsza godzina&2,80&1,60\\
			\hline
			3.&druga godzina&3,20&1,80\\
			\hline
			4.&trzecia godzina&3,60&2,00\\
			\hline
			5.&każda kolejna godzina&2,80&1,60\\
			\hline
		\end{tabular}
	\end{center}	
\end{table}

Ważna jest także informacja odnośnie czasu w którym bilety obwiązują, a mianowicie moment zakupu jest także chwilą, w której zaczynają być ważne. Nie ma możliwości zakupu biletów tzw. ``do przodu'', których czas rozpoczęcia jest późniejszy od daty zakupu \cite{sumowanie}. 

\subsection{Zakup i kontrola biletu}

Bilety postojowe są dostępne do kupienia w parkomatach, rozmieszczonych w miejscach gdzie obowiązuje strefa płatnego parkowania. Długość postoju ustalana jest na bazie kwoty, jaka została wpłacona oraz podstrefy, w której bilet został kupiony. Płatność może być realizowana gotówką, ale także przy użyciu karty SKA (Szczecińska Karta Aglomeracyjna), która w tym przypadku działa jak wirtualna portmonetka. Otrzymany wydruk potwierdzający opłacenie miejsca, należy umieścić za przednią szybą pojazdu. Kontroler sprawdza czy bilet został kupiony na odpowiednią podstrefę i czy nie przekroczono czasu postoju.

Oprócz parkomatów, od pewnego czasu w Szczecinie bilet może być kupiony także poprzez aplikację na urządzenia mobilne, a jednym z takich rozwiązań jest system moBiLET. Dostępny w wielu miastach Polski, pozwala na zakup nie tylko biletów postojowych, ale także komunikacji miejskiej, czy kolejowych. Po pobraniu aplikacji i zarejestrowaniu, niezbędne jest jeszcze doładowanie wirtualnej portmonetki, powiązanej z kontem użytkownika. W przypadku korzystania z strefy płatnego parkowania, konieczne jest jeszcze dodanie samochodu oraz umieszczenie w nim informacji, że bilet dla danego pojazdu został opłacony za pomocą tej właśnie aplikacji. Dzięki tej informacji, kontroler po wprowadzeniu numeru rejestracyjnego pojazdu, będzie mógł sprawdzić ważność biletu.

\subsubsection*{System ParQ}

Tworzony w ramach pracy system ParQ różni się od istniejących rozwiązań wykorzystaniem kodów QR. Każdy z pojazdów będzie musiał posiadać plakietkę z tym kodem, gdyż to za ich pomocą odbywa się identyfikacja pojazdów w systemie. Z perspektywy kierowcy, działanie tego systemu jest zbliżone do istniejących już rozwiązań. Po założeniu oraz doładowaniu konta odpowiednią kwotą, konieczne jest jeszcze dodanie swojego pojazdu. Po tym kroku kierowca otrzyma wiadomość e-mail z kodem QR, który powinien zostać umieszczony pod przednią szybą pojazdu. Od tego momentu wystarczy jedynie kupić bilet. 

Istotna różni się natomiast sposób przeprowadzania kontroli. Dzięki zastosowaniu kodów QR, nie ma potrzeby ręcznego wprowadzania numeru tablicy rejestracyjnej. Osoba sprawdzająca bilet będzie musiała jedynie zeskanować za pomocą aparatu wbudowanego w telefon plakietkę, znajdująca się pod przednią szybą pojazdu. Następnie informacja zwrócona z serwera tego systemu powiadomi go o tym, czy dany samochód ma opłacony bilet postojowy.
		\newpage
		
		\section{Projekt systemu}
\subsection{Wymagania funkcjonalne}
\subsection{Wymagania niefunkcjonalne}
\subsection{Diagramy UML}
\subsection{Projekt bazy danych}
		\newpage
		
		\section{Implementacja systemu}

\subsection{Wykorzystane narzędzia}
\subsection{Wyniki działania systemu}
		\newpage
		
		\section*{Podsumowanie}
\addcontentsline{toc}{section}{Podsumowanie}

Celem niniejszej pracy było stworzenie systemu płatności dla strefy płatnego parkowania. Do jego głównych zadań należało określenie wymagań systemu, zaimplementowanie serwera oraz aplikacji mobilnych. Użytkownicy korzystający z tego systemu mogą doładowywać swoje konta, rejestrować pojazdy oraz kupować bilety postojowe. Osoby kontrolujące mają możliwość sprawdzania ważności biletów postojowych. Główną cechą, która wyróżnia ten system spośród większości istniejących na rynku rozwiązań, jest zastosowanie kodów QR, za pomocą których identyfikowane są zaparkowane samochody. Postawiony w tej pracy cel został zrealizowany.
\\
\\
Pierwszy rozdział w całości poświęcony był płatnościom elektronicznym. Przedstawione w nim informacje pozwoliły na dokonanie analizy różnych metod płatności, pod kątem ich przydatności w systemach internetowych. Na bazie tych rozważań wybrana została forma zawierania transakcji, która najlepiej spełnia wymagania tworzonego systemu. W drugim rozdziale czytelnik został zapoznany z technologiami wykorzystywanymi do realizacji zadań pracy, a także pojęciami z nimi związanymi. Były one istotne dla pełnego zrozumienia dalszej części pracy. W rozdziale 3. omówiona została Strefa Parkowania w Szczecinie oraz dokonano porównania systemu ParQ z już istniejącymi rozwiązaniami, które spełniają podobne zadania. Kolejny rozdział zawiera dokumentację techniczną, natomiast ostatni, poświęcony został szczegółom implementacji.
\\
\\
Najbardziej pomocnymi pozycjami z bibliografii, były publikacje o tematyce dotyczącej płatności elektronicznych. Szczególnie przydatna okazała się książka autorstwa Artura Borcucha, pt. \textit{Pieniądz elektroniczny pieniądz przyszłości -- analiza ekonomiczno-prawna}. W niezwykle wyczerpujący sposób opisane zostały w niej poszczególne etapy rozwoju pieniądza elektronicznego, które omówiono w rozdziale pierwszym. Szczególnie ważna na etapie implementacji systemu była dokumentacja Django. Głównie dzięki przejrzystości oraz licznym przykładom w postaci fragmentów kodu, którymi opatrzone zostało prawie każde poruszane tam zagadnienie.
\\
\\
Mam nadzieję, że zebrane materiały okazały się wystarczające do zapoznania z zastosowanymi narzędziami oraz technologiami, użytymi do realizacji zadań pracy. Wybór tematu został podyktowany także rosnącą popularnością elektronicznych metod w internecie i aplikacjach mobilnych. Liczę, że podjęta analiza różnych form płatności, wykazała słuszność zastosowanej w tym systemie metody. 
		\newpage
		
		\section*{Słownik pojęć}
\addcontentsline{toc}{section}{Słownik pojęć}

\begin{description}
	\item[E-commerce] -- (e-handel, ang. handel elektroniczny) 
	\item[Płatności elektroniczne]
	\item[Dostawca usług płatniczych]
	\item[System płatniczy]
	\item[Instrument płatniczy]
	\item[Bankowość elektroniczna]
	\item[Bankowość internetowa]
	\item[Bankowość wirtualna]
	\item dispatcher
\end{description}
		\newpage
		
		\bibliographystyle{plplain}
		\bibliography{bibfile.bib}
		\addcontentsline{toc}{section}{Literatura} 
	\end{onehalfspacing}
\end{document}